\begin{resumo}[Abstract]
 \begin{otherlanguage*}{english}
 
This paper presents the importance of managing the maintenance of the asset’s organization with focus electronic equipment, which sometimes have their neglected maintenance, being seen as an expense source that does not have a noticeable return. Here are presented the basics of maintenance of these equipment management, bringing the main types of maintenance described in the literature and are also presented patterns and techniques that governing the asset management (ISO 5500x and PAS 55).
The foundation obtained in the literature shows that poor management of maintenance wastes resources, bringing extra and unplanned expenses, and also generates precarious and delayed solutions that increase the rate of unavailability of assets. To solve these problems the literature defines the main types of maintenance: corrective, preventive, predictive, and proactive, and states that there is no general solution to the problem, leaving the context and the task of nature in which these devices are applied to setting the best course of action, and the particular solution for each case.
So as object of study and validation, this research portrays the current maintenance management mode of electronic equipment used by the Directorate of Equipment Maintenance of the University of Brasilia (DIMEQ / UNB) and draws some of its processes. To then suggest better control techniques, planning and coordination to have assets that can generate value to the business by employing the solution concepts to Mean Time to Failure,  Mean Time to Repair,  Mean Time Between Failures, performance, reliability and cost. As well as suggest a solution to computerize these data constructed at work.
The survey defines maintenance performance indicators for the particular case of UNB, seeking to measure the moment of greatest probability of failure for electronic equipment and quantify the costs of maintenance. Thus helping the manager to know the best time to carry out maintenance or show comparative data between the maintenance costs or replacement of equipment. However, these indicators take into consideration the types of the grouping equipment into categories as there are devices which type of maintenance is recommended corrective, and others may be preventive, etc.


   \vspace{\onelineskip}
 
   \noindent 
   \textbf{Key-words}: maintenance, asset management, electronic equipment, UNB, maintenance management.
 \end{otherlanguage*}
\end{resumo}
