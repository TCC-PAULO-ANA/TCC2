\begin{apendicesenv}

\partapendices

\chapter{Especificação dos Casos de Uso}
\label{esp-uc}

%---------------------------------------------------------%

\begin{table}[H]
\centering
\caption{Caso de Uso - \textit{Visualizar Gráficos}. \textbf{Fonte: Autor.}}
\label{uc01}
\begin{tabular}{ | p{5cm} | p{10cm} |  }
\hline
	\textbf{Caso de uso 01} & Visualizar Gráficos \\ \hline
	\textbf{Descrição} & Sistema deve exibir um gráfico na tela principal do sistema (Figura~\ref{tela-principal}), com a situação de todos os equipamentos do sistema. E outro gráfico na tela de listagem dos equipamentos por especificação (Figura~\ref{tela-lista-eq}). Situação do equipamento pode \textbf{ser em uso, em manutenção e parado}. \\ \hline
	\textbf{Ator} & Gestor e Técnico \\ \hline
	\textbf{Pré-condições} & Já terem dados de equipamentos importados do SIPAT para o Sistema. \\ \hline
	\textbf{Pós-condições} & Visualização dos gráficos nas telas definidas. \\ \hline
	\textbf{Fluxos de eventos} & \begin{enumerate}
									\item Entrar no Sistema e visualizar gráfico com a situação de todos os equipamentos do sistema.       
									\item Visualizar Listagem dos equipamentos na Tela Principal (Figura~\ref{tela-principal}).
									\item Clicar no botão \textbf{Visualizar}
									\item Ser redirecionado para a Tela com a Listagem dos Equipamentos (Figura~\ref{tela-lista-eq}) de um determinada especificação.    
									\item Visualizar Gráfico da situação dos equipamentos da especificação (Figura~\ref{tela-lista-eq}).
								 \end{enumerate}   \\ \hline
	\textbf{Pontos de Extensão} & Nenhum. \\ \hline
\end{tabular}
\end{table}

%---------------------------------------------------------%

\begin{table}[H]
\centering
\caption{Caso de Uso - \textit{Visualizar Listagem com os equipamentos classificados por grupo e especificação}. \textbf{Fonte: Autor.}}
\label{uc02}
\begin{tabular}{ | p{5cm} | p{10cm} |  }
\hline
	\textbf{Caso de uso 02} & Visualizar Listagem com os equipamentos classificados por grupo e especificação. \\ \hline
	\textbf{Descrição} & Na tela principal do sistema (Figura~\ref{tela-principal}), deve aparecer uma lista, abaixo do gráfico com os dados dos equipamentos, agrupados por sua especificação. Campos: \begin{itemize}
															                    \item Grupo 
															                    \item Especificação 
															                    \item Quantidade 
															                    \item Em Manutenção 
															                    \item Em uso
															                    \item Parados 
															                    \item Botão \textbf{visualizar}. 
															                    \end{itemize} \\ \hline
	\textbf{Ator} & Gestor e Técnico \\ \hline
	\textbf{Pré-condições} & Já terem dados de equipamentos importados do SIPAT para o Sistema. \\ \hline
	\textbf{Pós-condições} & Visualização da listagem dos equipamentos com os campos definidos.. \\ \hline
	\textbf{Fluxos de eventos} & \begin{enumerate}
									\item Entrar no Sistema e na tela inicial visualizar a lista (Figura~\ref{tela-principal}).    
									\item Lista deve estar abaixo do gráfico.
									\item Ao clicar no botão \textbf{Visualizar}, ator deve ser redirecionado para tela da especificação (Figura~\ref{tela-lista-eq}).   
								 \end{enumerate}   \\ \hline
	\textbf{Pontos de Extensão} & Nenhum. \\ \hline
\end{tabular}
\end{table}

%---------------------------------------------------------%

\begin{table}[H]
\centering
\caption{Caso de Uso - \textit{Visualizar Listagem dos equipamentos de uma especificação}. \textbf{Fonte: Autor.}}
\label{uc03}
\begin{tabular}{ | p{5cm} | p{10cm} |  }
\hline
	\textbf{Caso de uso 03} & Visualizar Listagem dos equipamentos de uma especificação. \\ \hline
	\textbf{Descrição} & Na tela de uma especificação de equipamentos (Figura~\ref{tela-lista-eq}), deve aparecer uma lista, abaixo do gráfico (especificado no caso de uso 01) com os dados dos equipamentos. Campos: \begin{itemize}
															                    \item Registro
															                    \item Situação (em uso, em manutenção, parado) 
															                    \item Próxima Manutenção 
															                    \item Manutenções Realizadas 
															                    \item Tipo de Manutenção
															                    \item Botão \textbf{visualizar}. 
															                    \end{itemize} \\ \hline
	\textbf{Ator} & Gestor e Técnico \\ \hline
	\textbf{Pré-condições} & Clicar no botão visualizar na página principal. \\ \hline
	\textbf{Pós-condições} & Visualização da listagem dos equipamentos com os campos definidos. \\ \hline
	\textbf{Fluxos de eventos} & \begin{enumerate}
									\item Entrar no Sistema e na tela inicial visualizar a lista de Equipamentos (Figura~\ref{tela-principal}).    
									\item Clicar no botão \textbf{Visualizar} e ser redirecionado para tela da especificação (Figura~\ref{tela-lista-eq}).  
							        \item Lista deve estar abaixo do gráfico da tela.
							        \item Ao clicar no botão \textbf{visualizar}, da listagem, o ator deve ser redirecionado para tela de menu do equipamento (Figura~\ref{tela-menu-eq}).
								 \end{enumerate}   \\ \hline
	\textbf{Pontos de Extensão} & Nenhum. \\ \hline
\end{tabular}
\end{table}

%---------------------------------------------------------%

\begin{table}[H]
\centering
\caption{Caso de Uso - \textit{Visualizar Menu do Equipamento}. \textbf{Fonte: Autor.}}
\label{uc04}
\begin{tabular}{ | p{5cm} | p{10cm} |  }
\hline
	\textbf{Caso de uso 04} & Visualizar Menu do Equipamento. \\ \hline
	\textbf{Descrição} & Ao clicar em visualizar, em um equipamento, deve ser mostrado uma página com o menu dele. Ícones: \begin{itemize}
															                    \item Dados Cadastrais
															                    \item Ordens de Serviço 
															                    \item Tipo de Manutenção 
															                    \item Indicadores 
															                    \item Componentes Principais
															                    \end{itemize} \\ \hline
	\textbf{Ator} & Técnico \\ \hline
	\textbf{Pré-condições} & Clicar no botão visualizar na página de especificação do equipamento. \\ \hline
	\textbf{Pós-condições} & Visualização da Tela de Menu dos equipamentos. \\ \hline
	\textbf{Fluxos de eventos} & \begin{enumerate}
									\item Clicar em visualizar em um equipamento da lista (Figura~\ref{tela-lista-eq}).    
									\item Visualizar os campos da tela (Figura~\ref{tela-menu-eq})
							        \item Clicar em um ícone e ser direcionado para sua tela.
								 \end{enumerate}   \\ \hline
	\textbf{Pontos de Extensão} & Nenhum. \\ \hline
\end{tabular}
\end{table}

%---------------------------------------------------------%

\begin{table}[H]
\centering
\caption{Caso de Uso - \textit{Visualizar Dados Cadastrais do Equipamento}. \textbf{Fonte: Autor.}}
\label{uc05}
\begin{tabular}{ | p{5cm} | p{10cm} |  }
\hline
	\textbf{Caso de uso 05} & Visualizar Dados Cadastrais do Equipamento. \\ \hline
	\textbf{Descrição} & Ao clicar em dados cadastrais, na tela do equipamento, deve ser mostrado uma página com os dados. Os dados serão importados do SIPAT. \\ \hline
	\textbf{Ator} & Técnico \\ \hline
	\textbf{Pré-condições} & Nenhuma. \\ \hline
	\textbf{Pós-condições} & Visualização dados de cadastro do equipamento. \\ \hline
	\textbf{Fluxos de eventos} & \begin{enumerate}
									\item Clicar em Dados Cadastrais (Figura~\ref{tela-menu-eq}).    
									\item Ser redirecionado para tela de dados do equipamento (Figura~\ref{tela-dados-cad}).
								 \end{enumerate}   \\ \hline
	\textbf{Pontos de Extensão} & Nenhum. \\ \hline
\end{tabular}
\end{table}

%---------------------------------------------------------%

\begin{table}[H]
\centering
\caption{Caso de Uso - \textit{Visualizar Dados Cadastrais do Equipamento}. \textbf{Fonte: Autor.}}
\label{uc06}
\begin{tabular}{ | p{5cm} | p{10cm} |  }
\hline
	\textbf{Caso de uso 06} & Visualizar Dados Cadastrais do Equipamento. \\ \hline
	\textbf{Descrição} & Ao clicar em dados cadastrais, na tela do equipamento, deve ser mostrado uma página com os dados. Os dados serão importados do SIPAT. \\ \hline
	\textbf{Ator} & Técnico \\ \hline
	\textbf{Pré-condições} & Nenhuma. \\ \hline
	\textbf{Pós-condições} & Visualização dados de cadastro do equipamento. \\ \hline
	\textbf{Fluxos de eventos} & \begin{enumerate}
									\item Clicar em Dados Cadastrais (Figura~\ref{tela-menu-eq}).    
									\item Ser redirecionado para tela de dados do equipamento (Figura~\ref{tela-dados-cad}).
								 \end{enumerate}   \\ \hline
	\textbf{Pontos de Extensão} & Nenhum. \\ \hline
\end{tabular}
\end{table}

%---------------------------------------------------------%
%-------------------------------------------------------------------------------------------------------------------------------%

\chapter{Protótipos das Telas do Sistema}

\graphicspath{{figuras/prototipos/}}
\begin{figure}[H]
\centering
\includegraphics[width=1.0\textwidth]{tela-principal}
\caption{Protótipo da Tela Principal do Sistema. \textbf{Fonte: Autor.}}
\label{tela-principal}
\end{figure}

%---------------------------------------------------------%

\graphicspath{{figuras/prototipos/}}
\begin{figure}[H]
\centering
\includegraphics[width=1.0\textwidth]{tela-importar-equipamentos}
\caption{Protótipo da Tela para Importar os Equipamentos Cadastrados no SIPAT. \textbf{Fonte: Autor.}}
\label{tela-importar-equipamentos}
\end{figure}

%---------------------------------------------------------%

\graphicspath{{figuras/prototipos/}}
\begin{figure}[H]
\centering
\includegraphics[width=1.0\textwidth]{tela-rel-gastos}
\caption{Protótipo da Tela do Relatório dos Gastos realizados com manutenções. \textbf{Fonte: Autor.}}
\label{tela-rel-gastos}
\end{figure}

\graphicspath{{figuras/prototipos/}}
\begin{figure}[H]
\centering
\includegraphics[width=1.0\textwidth]{tela-rel-os}
\caption{Protótipo da Tela do Relatório de Ordens de Serviço. \textbf{Fonte: Autor.}}
\label{tela-rel-os}
\end{figure}

%---------------------------------------------------------%

\graphicspath{{figuras/prototipos/}}
\begin{figure}[H]
\centering
\includegraphics[width=1.0\textwidth]{tela-prox-man}
\caption{Protótipo da Tela das Manutenções a serem realizadas. \textbf{Fonte: Autor.}}
\label{tela-prox-man}
\end{figure}

%---------------------------------------------------------%


\graphicspath{{figuras/prototipos/}}
\begin{figure}[H]
\centering
\includegraphics[width=1.0\textwidth]{tela-abrir-os}
\caption{Protótipo da Tela para Abrir uma Nova Ordem de Serviço. \textbf{Fonte: Autor.}}
\label{tela-abrir-os}
\end{figure}

%---------------------------------------------------------%

\graphicspath{{figuras/prototipos/}}
\begin{figure}[H]
\centering
\includegraphics[width=1.0\textwidth]{tela-lista-eq}
\caption{Protótipo da Tela da Listagem de Todos os Equipamentos de uma Especificação. \textbf{Fonte: Autor.}}
\label{tela-lista-eq}
\end{figure}

%---------------------------------------------------------%


\graphicspath{{figuras/prototipos/}}
\begin{figure}[H]
\centering
\includegraphics[width=1.0\textwidth]{tela-menu-eq}
\caption{Protótipo da Tela com o Menu do Equipamento. \textbf{Fonte: Autor.}}
\label{tela-menu-eq}
\end{figure}

%---------------------------------------------------------%

\graphicspath{{figuras/prototipos/}}
\begin{figure}[H]
\centering
\includegraphics[width=1.0\textwidth]{tela-dados-cad}
\caption{Protótipo da Tela dos Dados do Equipamento Cadastrados pelo SIPAT. \textbf{Fonte: Autor.}}
\label{tela-dados-cad}
\end{figure}


%---------------------------------------------------------%

Tela - Indicadores do Equipamento



%---------------------------------------------------------%


\graphicspath{{figuras/prototipos/}}
\begin{figure}[H]
\centering
\includegraphics[width=1.0\textwidth]{tela-comp-princ}
\caption{Protótipo da Tela para Adicionar os Componentes Principais do Equipamento. \textbf{Fonte: Autor.}}
\label{tela-comp-princ}
\end{figure}

%---------------------------------------------------------%


\graphicspath{{figuras/prototipos/}}
\begin{figure}[H]
\centering
\includegraphics[width=1.0\textwidth]{tela-tipo-man}
\caption{Protótipo da Tela para Escolha do Tipo de Manutenção que será realizada no Equipamento. \textbf{Fonte: Autor.}}
\label{tela-tipo-man}
\end{figure}

%---------------------------------------------------------%


\graphicspath{{figuras/prototipos/}}
\begin{figure}[H]
\centering
\includegraphics[width=1.0\textwidth]{tela-os-eq}
\caption{Protótipo da Tela de Listagem das Ordens de Serviço já Abertas para o Equipamento. \textbf{Fonte: Autor.}}
\label{tela-os-eq}
\end{figure}

\end{apendicesenv}