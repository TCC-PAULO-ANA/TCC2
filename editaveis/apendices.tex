\begin{apendicesenv}

\partapendices

\chapter{Especificação dos Casos de Uso}
\label{esp-uc}

%---------------------------------------------------------%

\begin{table}[H]
\centering
\caption{Caso de Uso - \textit{Visualizar Gráficos}. \textbf{Fonte: Autor.}}
\label{uc01}
\begin{tabular}{ | p{5cm} | p{10cm} |  }
\hline
	\textbf{Caso de uso 01} & Visualizar Gráficos \\ \hline
	\textbf{Descrição} & Sistema deve exibir um gráfico na tela principal do sistema (Figura~\ref{tela-principal}), com a situação de todos os equipamentos do sistema. E outro gráfico na tela de listagem dos equipamentos por especificação (Figura~\ref{tela-lista-eq}). Situação do equipamento pode \textbf{ser em uso, em manutenção e parado}. \\ \hline
	\textbf{Ator} & Gestor e Técnico \\ \hline
	\textbf{Pré-condições} & Já terem dados de equipamentos importados do SIPAT para o Sistema. \\ \hline
	\textbf{Pós-condições} & Visualização dos gráficos nas telas definidas. \\ \hline
	\textbf{Fluxos de eventos} & \begin{enumerate}
									\item Entrar no Sistema e visualizar gráfico com a situação de todos os equipamentos do sistema.       
									\item Visualizar Listagem dos equipamentos na Tela Principal (Figura~\ref{tela-principal}).
									\item Clicar no botão \textbf{Visualizar}
									\item Ser redirecionado para a Tela com a Listagem dos Equipamentos (Figura~\ref{tela-lista-eq}) de um determinada especificação.    
									\item Visualizar Gráfico da situação dos equipamentos da especificação (Figura~\ref{tela-lista-eq}).
								 \end{enumerate}   \\ \hline
	\textbf{Pontos de Extensão} & Nenhum. \\ \hline
\end{tabular}
\end{table}

%---------------------------------------------------------%

\begin{table}[H]
\centering
\caption{Caso de Uso - \textit{Visualizar Listagem com os equipamentos classificados por grupo e especificação}. \textbf{Fonte: Autor.}}
\label{uc02}
\begin{tabular}{ | p{5cm} | p{10cm} |  }
\hline
	\textbf{Caso de uso 02} & Visualizar Listagem com os equipamentos classificados por grupo e especificação. \\ \hline
	\textbf{Descrição} & Na tela principal do sistema (Figura~\ref{tela-principal}), deve aparecer uma lista, abaixo do gráfico com os dados dos equipamentos, agrupados por sua especificação. Campos: \begin{itemize}
															                    \item Grupo 
															                    \item Especificação 
															                    \item Quantidade 
															                    \item Em Manutenção 
															                    \item Em uso
															                    \item Parados 
															                    \item Botão \textbf{visualizar}. 
															                    \end{itemize} \\ \hline
	\textbf{Ator} & Gestor e Técnico \\ \hline
	\textbf{Pré-condições} & Já terem dados de equipamentos importados do SIPAT para o Sistema. \\ \hline
	\textbf{Pós-condições} & Visualização da listagem dos equipamentos com os campos definidos.. \\ \hline
	\textbf{Fluxos de eventos} & \begin{enumerate}
									\item Entrar no Sistema e na tela inicial visualizar a lista (Figura~\ref{tela-principal}).    
									\item Lista deve estar abaixo do gráfico.
									\item Ao clicar no botão \textbf{Visualizar}, ator deve ser redirecionado para tela da especificação (Figura~\ref{tela-lista-eq}).   
								 \end{enumerate}   \\ \hline
	\textbf{Pontos de Extensão} & Nenhum. \\ \hline
\end{tabular}
\end{table}

%---------------------------------------------------------%

\begin{table}[H]
\centering
\caption{Caso de Uso - \textit{Visualizar Listagem dos equipamentos de uma especificação}. \textbf{Fonte: Autor.}}
\label{uc03}
\begin{tabular}{ | p{5cm} | p{10cm} |  }
\hline
	\textbf{Caso de uso 03} & Visualizar Listagem dos equipamentos de uma especificação. \\ \hline
	\textbf{Descrição} & Na tela de uma especificação de equipamentos (Figura~\ref{tela-lista-eq}), deve aparecer uma lista, abaixo do gráfico (especificado no caso de uso 01) com os dados dos equipamentos. Campos: \begin{itemize}
															                    \item Registro
															                    \item Situação (em uso, em manutenção, parado) 
															                    \item Próxima Manutenção 
															                    \item Manutenções Realizadas 
															                    \item Tipo de Manutenção
															                    \item Botão \textbf{visualizar}. 
															                    \end{itemize} \\ \hline
	\textbf{Ator} & Gestor e Técnico \\ \hline
	\textbf{Pré-condições} & Clicar no botão visualizar na página principal. \\ \hline
	\textbf{Pós-condições} & Visualização da listagem dos equipamentos com os campos definidos. \\ \hline
	\textbf{Fluxos de eventos} & \begin{enumerate}
									\item Entrar no Sistema e na tela inicial visualizar a lista de Equipamentos (Figura~\ref{tela-principal}).    
									\item Clicar no botão \textbf{Visualizar} e ser redirecionado para tela da especificação (Figura~\ref{tela-lista-eq}).  
							        \item Lista deve estar abaixo do gráfico da tela.
							        \item Ao clicar no botão \textbf{visualizar}, da listagem, o ator deve ser redirecionado para tela de menu do equipamento (Figura~\ref{tela-menu-eq}).
								 \end{enumerate}   \\ \hline
	\textbf{Pontos de Extensão} & Nenhum. \\ \hline
\end{tabular}
\end{table}

%---------------------------------------------------------%

\begin{table}[H]
\centering
\caption{Caso de Uso - \textit{Visualizar Menu do Equipamento}. \textbf{Fonte: Autor.}}
\label{uc04}
\begin{tabular}{ | p{5cm} | p{10cm} |  }
\hline
	\textbf{Caso de uso 04} & Visualizar Menu do Equipamento. \\ \hline
	\textbf{Descrição} & Ao clicar em visualizar, em um equipamento, deve ser mostrado uma página com o menu dele. Ícones: \begin{itemize}
															                    \item Dados Cadastrais
															                    \item Ordens de Serviço 
															                    \item Tipo de Manutenção 
															                    \item Indicadores 
															                    \item Componentes Principais
															                    \end{itemize} \\ \hline
	\textbf{Ator} & Técnico \\ \hline
	\textbf{Pré-condições} & Clicar no botão visualizar na página de especificação do equipamento. \\ \hline
	\textbf{Pós-condições} & Visualização da Tela de Menu dos equipamentos. \\ \hline
	\textbf{Fluxos de eventos} & \begin{enumerate}
									\item Clicar em visualizar em um equipamento da lista (Figura~\ref{tela-lista-eq}).    
									\item Visualizar os campos da tela (Figura~\ref{tela-menu-eq})
							        \item Clicar em um ícone e ser direcionado para sua tela.
								 \end{enumerate}   \\ \hline
	\textbf{Pontos de Extensão} & Nenhum. \\ \hline
\end{tabular}
\end{table}

%---------------------------------------------------------%

\begin{table}[H]
\centering
\caption{Caso de Uso - \textit{Visualizar Dados Cadastrais do Equipamento}. \textbf{Fonte: Autor.}}
\label{uc05}
\begin{tabular}{ | p{5cm} | p{10cm} |  }
\hline
	\textbf{Caso de uso 05} & Visualizar Dados Cadastrais do Equipamento. \\ \hline
	\textbf{Descrição} & Ao clicar em dados cadastrais, na tela do equipamento, deve ser mostrado uma página com os dados. Os dados serão importados do SIPAT. \\ \hline
	\textbf{Ator} & Técnico \\ \hline
	\textbf{Pré-condições} & Nenhuma. \\ \hline
	\textbf{Pós-condições} & Visualização dados de cadastro do equipamento. \\ \hline
	\textbf{Fluxos de eventos} & \begin{enumerate}
									\item Clicar em Dados Cadastrais (Figura~\ref{tela-menu-eq}).    
									\item Ser redirecionado para tela de dados do equipamento (Figura~\ref{tela-dados-cad}).
								 \end{enumerate}   \\ \hline
	\textbf{Pontos de Extensão} & Nenhum. \\ \hline
\end{tabular}
\end{table}

%---------------------------------------------------------%

\begin{table}[H]
\centering
\caption{Caso de Uso - \textit{Adicionar Componentes Principais do Equipamento}. \textbf{Fonte: Autor.}}
\label{uc06}
\begin{tabular}{ | p{5cm} | p{10cm} |  }
\hline
	\textbf{Caso de uso 06} & Adicionar Componentes Principais do Equipamento. \\ \hline
	\textbf{Descrição} & Ao clicar em Componentes Adicionais, na tela do equipamento, deve ser mostrada uma página para serem adicionados componentes do equipamento, que precisão ser observados, na manutenção preventiva. Campos: \begin{itemize}
															                    \item Adicionar (botão)
															                    \item Nome
															                    \item Departamento 
															                    \item Estado 
															                    \item observações
															                    \item Editar (botão)
															                    \end{itemize} \\ \hline
	\textbf{Ator} & Técnico \\ \hline
	\textbf{Pré-condições} & estar na tela do equipamento. \\ \hline
	\textbf{Pós-condições} & Ter componente adicionado na listagem. \\ \hline
	\textbf{Fluxos de eventos} & \begin{enumerate}
									\item Clicar em Componentes Principais (Figura~\ref{tela-menu-eq}).    
									\item Ser redirecionado para tela de adicionar o componente. (Figura~\ref{tela-comp-princ}).
									\item Inserir nome do componente.
									\item Clicar em Adicionar.
									\item Visualizar componente adicionado na listagem.
									\item Listagem deve conter os campos (nome, departamento, estado, observações).
									\item Conseguir editar as informações.
									\item Esses dados da listagem, devem ser puxados da ordem de serviço aberta para manutenção.
								 \end{enumerate}   \\ \hline
	\textbf{Pontos de Extensão} & Nenhum. \\ \hline
\end{tabular}
\end{table}

%---------------------------------------------------------%

\begin{table}[H]
\centering
\caption{Caso de Uso - \textit{Manter Ordens de Serviço}. \textbf{Fonte: Autor.}}
\label{uc07}
\begin{tabular}{ | p{5cm} | p{10cm} |  }
\hline
	\textbf{Caso de uso 07} & Manter Ordens de Serviço. \\ \hline
	\textbf{Descrição} & Ao clicar em Ordens de Serviço, o técnico deverá ser redirecionado para uma página que terá uma lista com todas as ordens de serviços abertas para o equipamento e deve conseguir abrir uma nova ordem de serviço. Na ordem de serviço devem existir os campos para logar as horas trabalhadas e adicionar custos de peças para substituição. Campos: \begin{itemize}
															                    \item Abrir Nova Ordem de Serviço (botão)
															                    \item Aberta por
															                    \item Departamento 
															                    \item Data de Início 
															                    \item Data de Fechamento
															                    \item Situação
															                    \item ver (botão)
															                    \end{itemize} \\ \hline
	\textbf{Ator} & Técnico \\ \hline
	\textbf{Pré-condições} & estar na tela do equipamento. \\ \hline
	\textbf{Pós-condições} & Ver listagem, conseguir abrir nova ordem de serviço e visualizar as da listagem. \\ \hline
	\textbf{Fluxos de eventos} & \begin{enumerate}
									\item Clicar em Ordens de Serviço (Figura~\ref{tela-menu-eq}).    
									\item Ser redirecionado para tela de ordens de Serviço. (Figura~\ref{tela-os-eq}).
									\item Clicar em Abrir Nova Ordem de Serviço.
									\item Ser redicionado para página (Figura~\ref{tela-abrir-os}.
									\item Visualizar campos para o preenchimento da ordem de serviço.
									\item Listagem deve conter os campos citados na descrição.
									\item Conseguir visualizar ordem de serviço.
									\item Esses dados da listagem, devem ser puxados da ordem de serviço aberta para manutenção.
								 \end{enumerate}   \\ \hline
	\textbf{Pontos de Extensão} & Nenhum. \\ \hline
\end{tabular}
\end{table}

%---------------------------------------------------------%

\begin{table}[H]
\centering
\caption{Caso de Uso - \textit{Escolher Tipo de Manutenção}. \textbf{Fonte: Autor.}}
\label{uc08}
\begin{tabular}{ | p{3cm} | p{13cm} |  }
\hline
	\textbf{Caso de uso 08} & Escolher Tipo de Manutenção. \\ \hline
	\textbf{Descrição} & Técnico conseguir escolher qual tipo de manutenção será realizada no equipamento. O tipo poderá ser escolhido por equipamento (individualmente) ou para um lote de equipamentos (por especificação). A escolha por lote, será feita, por meio de um questionário, que sugerir a melhor estratégia para o tipo de equipamento. O tipo padrão de manutenção é a corretiva. Ou seja, assim que os equipamentos são importados, eles são classificados com esse tipo, pra mudar, precisa ir para a tela de tipo de manutenção. Campos: \begin{itemize}
															                    \item Preventiva
															                    \item Corretiva
															                    \item Período de realização da manutenção
															                    \item Data Inicial 
															                    \item Escolher data da compra
															                    \item A próxima manutenção será em
															                    \item Aplicar (botão)
															                    \item Escolher Tipo de Manutenção (botão)
															                    \item Agendar (botão)
															                    \end{itemize} \\ \hline
	\textbf{Ator} & Técnico \\ \hline
	\textbf{Pré-condições} & estar na tela do equipamento. \\ \hline
	\textbf{Pós-condições} & Escolher tipo de manutenção. \\ \hline
	\textbf{Fluxos de eventos} & \begin{enumerate}
									\item Clicar em Tipo de Manutenção (Figura~\ref{tela-menu-eq}).    
									\item Visualizar os tipos Corretiva e Preventiva.
									\item Selecionar Corretiva e clicar em aplicar.
									\item Selecionar preventiva, preencher os campos e clicar em aplicar.
									\item Escolher tipo por lote: entrar na tela da listagem de equipamentos por especificação (Figura~\ref{tela-lista-eq}).
									\item Clicar no botão Escolher Tipo de Manutenção.
									\item Ser redirecionado para tela (Figura~\ref{tela-quest-man})
									\item Se manutenção sugerida for a preventiva, clicar em agendar.
									\item Ser redirecionado para tela de tipo de manutenção (Figura~\ref{tela-tipo-man}).
									\item Preencher período e data de início.
									\item Se tipo de manutenção sugerido for corretiva, clicar em Aplicar.
								 \end{enumerate}   \\ \hline
\end{tabular}
\end{table}

%---------------------------------------------------------%

\begin{table}[H]
\centering
\caption{Caso de Uso - \textit{Visualizar Indicadores}. \textbf{Fonte: Autor.}}
\label{uc09}
\begin{tabular}{ | p{5cm} | p{10cm} |  }
\hline
	\textbf{Caso de uso 09} & Visualizar Indicadores. \\ \hline
	\textbf{Descrição} & O sistema mostrará para o técnico e gestor, indicadores de custo e de taxa de falhas de cada equipamento. O custo será calculado por meio das horas logadas pelo técnico na OS aberta para o reparo e também por custo de peças que precisarem ser substituídas. A taxa de falhas usará os conceitos de TMF, TMEF e TMR, para mostrar confiabilidade, desempenho e confiabilidade do equipamento. Serão mostrados gráficos para um acompanhamento visual da progressão dos indicadores. Campos: \begin{itemize}
															                    \item Preço do Equipamento
															                    \item Custo máximo das manutenções
															                    \item Custo da manutenção está em:
															                    \item Tempo Médio de Falha 
															                    \item Tempo Médio entre Falhas
															                    \item Tempo Médio de Reparo
															                    \item Gráficos de acompanhamento
															                    \end{itemize} \\ \hline
	\textbf{Ator} & Técnico e Gestor \\ \hline
	\textbf{Pré-condições} & estar na tela do equipamento. \\ \hline
	\textbf{Pós-condições} & Visualizar campos de custo e taxa de falha. \\ \hline
	\textbf{Fluxos de eventos} & \begin{enumerate}
									\item Clicar em Indeicadores (Figura~\ref{tela-menu-eq}).    								
									\item Ser redirecionado para tela de Tipo de Manutenção. (Figura~\ref{tela-indicador}).
									\item Visualizar Campos e gráfico de custo e taxa de falha.
								 \end{enumerate}   \\ \hline
	\textbf{Pontos de Extensão} & Nenhum. \\ \hline
\end{tabular}
\end{table}

%---------------------------------------------------------%

\begin{table}[H]
\centering
\caption{Caso de Uso - \textit{Importar Equipamentos do SIPAT}. \textbf{Fonte: Autor.}}
\label{uc10}
\begin{tabular}{ | p{5cm} | p{10cm} |  }
\hline
	\textbf{Caso de uso 10} & Importar Equipamentos do SIPAT. \\ \hline
	\textbf{Descrição} & Técnico conseguir visualizar equipamentos a serem importados e atualizar a lista de equipamento. Campos: \begin{itemize}
															                    \item Grupo
															                    \item Especificação
															                    \item Departamento
															                    \item Incluir 
															                    \item Atualizar Lista (botão)
															                    \item Importar (botão)
															                    \end{itemize} \\ \hline
	\textbf{Ator} & Técnico \\ \hline
	\textbf{Pré-condições} & Nenhuma \\ \hline
	\textbf{Pós-condições} & Conseguir importar equipamentos cadastrados no SIPAT para o sistema de manutenção. \\ \hline
	\textbf{Fluxos de eventos} & \begin{enumerate}
									\item Clicar em Importar Equipamentos (menu fixo no topo das telas).    
									\item Ser redirecionado para tela de importação. (Figura~\ref{tela-importar-equipamentos}).
									\item Visualizar listagem com os equipamentos disponíveis para importação.
									\item Clicar em importar.
									\item Conseguir atualizar a lista.
									\item Conseguir incluir todos os equipamentos de uma vez só.
									\item Conseguir selecionar quais equipamentos quer importar.
								 \end{enumerate}   \\ \hline
	\textbf{Pontos de Extensão} & Nenhum. \\ \hline
\end{tabular}
\end{table}

%---------------------------------------------------------%

\begin{table}[H]
\centering
\caption{Caso de Uso - \textit{Visualizar Menu no topo das telas}. \textbf{Fonte: Autor.}}
\label{uc11}
\begin{tabular}{ | p{5cm} | p{10cm} |  }
\hline
	\textbf{Caso de uso 11} & Visualizar Menu no topo das telas. \\ \hline
	\textbf{Descrição} & Deve ser visualizado uma barra de menu fixa em todas as telas do sistema. Campos: \begin{itemize}
															                    \item Importar Equipamentos
															                    \item Relatórios
															                    \item Próximas Manutenções
															                    \item Abrir Nova Ordem de Serviço
															                    \end{itemize} \\ \hline
	\textbf{Ator} & Técnico e Gestor \\ \hline
	\textbf{Pré-condições} & Nenhuma \\ \hline
	\textbf{Pós-condições} & Visualizar o menu fixo em todas as telas do sistema. \\ \hline
	\textbf{Fluxos de eventos} & \begin{enumerate}
									\item Entrar no sistema.  
									\item Navegar pelas telas e visualizar o menu em todas elas.
									\item Visualizar os campos especificados na descrição.
								 \end{enumerate}   \\ \hline
	\textbf{Pontos de Extensão} & Nenhum. \\ \hline
\end{tabular}
\end{table}

%---------------------------------------------------------%

\begin{table}[H]
\centering
\caption{Caso de Uso - \textit{Visualizar Relatórios}. \textbf{Fonte: Autor.}}
\label{uc12}
\begin{tabular}{ | p{5cm} | p{10cm} |  }
\hline
	\textbf{Caso de uso 12} & Visualizar Relatórios. \\ \hline
	\textbf{Descrição} & Visualizar relatório de gastos, que pode ser mensal e anual. Relatório de ordens de serviço, que mostrará as abertas e fechadas por mês e ano. Campos: \begin{itemize}
															                    \item Tipo de relatório (gastos totais, ordens de serviço)
															                    \item período (mensal e anual)
															                    \item gastos com manutenções preventivas
															                    \item gastos com manutenções corretivas
															                    \item total de gastos
															                    \item ordens de serviço abertas 
															                    \item ordens de serviço fechadas
															                    \end{itemize} \\ \hline
	\textbf{Ator} & Técnico e Gestor \\ \hline
	\textbf{Pré-condições} & Nenhuma \\ \hline
	\textbf{Pós-condições} & Visualizar relatórios. \\ \hline
	\textbf{Fluxos de eventos} & \begin{enumerate}
									\item Clicar em Relatório (menu fixo no topo das telas).
									\item ser redirecionado para  tela de relatórios (Figura~\ref{tela-rel-os} e Figura~\ref{tela-rel-gastos}) 
									\item Conseguir filtrar tipo de relatório.
									\item Só aparecer na tela, os campos inseridos nos filtros.
								 \end{enumerate}   \\ \hline
	\textbf{Pontos de Extensão} & Nenhum. \\ \hline
\end{tabular}
\end{table}

%---------------------------------------------------------%

\begin{table}[H]
\centering
\caption{Caso de Uso - \textit{Visualizar lista com manutenções a serem feitas}. \textbf{Fonte: Autor.}}
\label{uc13}
\begin{tabular}{ | p{5cm} | p{10cm} |  }
\hline
	\textbf{Caso de uso 13} & Visualizar lista com manutenções a serem feitas. \\ \hline
	\textbf{Descrição} & Mostrar listagem com todas as manutenções agendadas e corretivas da data d corrente para frente. A lista deve ser atualizada diariamente, retirando sempre as manutenções que já foram realizadas. Campos: \begin{itemize}
															                    \item Registro
															                    \item Especificação
															                    \item Departamento
															                    \item Tipo
															                    \item Data
															                    \item Prioridade 
															                    \end{itemize} \\ \hline
	\textbf{Ator} & Técnico e Gestor \\ \hline
	\textbf{Pré-condições} & Nenhuma \\ \hline
	\textbf{Pós-condições} & Visualizar relatórios. \\ \hline
	\textbf{Fluxos de eventos} & \begin{enumerate}
									\item Clicar em Próximas Manutenções (menu fixo no topo das telas).
									\item Ser redirecionado para  tela da listagem (Figura~\ref{tela-prox-man}).
									\item Visualizar listagem com campos da descrição.
								 \end{enumerate}   \\ \hline
	\textbf{Pontos de Extensão} & Nenhum. \\ \hline
\end{tabular}
\end{table}


%-------------------------------------------------------------------------------------------------------------------------------%

\chapter{Protótipos das Telas do Sistema}

\graphicspath{{figuras/prototipos/}}
\begin{figure}[H]
\centering
\includegraphics[width=1.0\textwidth]{tela-principal}
\caption{Protótipo da Tela Principal do Sistema. \textbf{Fonte: Autor.}}
\label{tela-principal}
\end{figure}

%---------------------------------------------------------%

\graphicspath{{figuras/prototipos/}}
\begin{figure}[H]
\centering
\includegraphics[width=1.0\textwidth]{tela-importar-equipamentos}
\caption{Protótipo da Tela para Importar os Equipamentos Cadastrados no SIPAT. \textbf{Fonte: Autor.}}
\label{tela-importar-equipamentos}
\end{figure}

%---------------------------------------------------------%
\pagebreak

\graphicspath{{figuras/prototipos/}}
\begin{figure}[H]
\centering
\includegraphics[width=1.0\textwidth]{tela-rel-gastos}
\caption{Protótipo da Tela do Relatório dos Gastos realizados com manutenções. \textbf{Fonte: Autor.}}
\label{tela-rel-gastos}
\end{figure}

\pagebreak

\graphicspath{{figuras/prototipos/}}
\begin{figure}[H]
\centering
\includegraphics[width=1.0\textwidth]{tela-rel-os}
\caption{Protótipo da Tela do Relatório de Ordens de Serviço. \textbf{Fonte: Autor.}}
\label{tela-rel-os}
\end{figure}

%---------------------------------------------------------%

\graphicspath{{figuras/prototipos/}}
\begin{figure}[H]
\centering
\includegraphics[width=1.0\textwidth]{tela-prox-man}
\caption{Protótipo da Tela das Manutenções a serem realizadas. \textbf{Fonte: Autor.}}
\label{tela-prox-man}
\end{figure}

%---------------------------------------------------------%


\graphicspath{{figuras/prototipos/}}
\begin{figure}[H]
\centering
\includegraphics[width=1.0\textwidth]{tela-abrir-os}
\caption{Protótipo da Tela para Abrir uma Nova Ordem de Serviço. \textbf{Fonte: Autor.}}
\label{tela-abrir-os}
\end{figure}

%---------------------------------------------------------%

\graphicspath{{figuras/prototipos/}}
\begin{figure}[H]
\centering
\includegraphics[width=1.0\textwidth]{tela-lista-eq}
\caption{Protótipo da Tela da Listagem de Todos os Equipamentos de uma Especificação. \textbf{Fonte: Autor.}}
\label{tela-lista-eq}
\end{figure}

%---------------------------------------------------------%


\graphicspath{{figuras/prototipos/}}
\begin{figure}[H]
\centering
\includegraphics[width=1.0\textwidth]{tela-menu-eq}
\caption{Protótipo da Tela com o Menu do Equipamento. \textbf{Fonte: Autor.}}
\label{tela-menu-eq}
\end{figure}

%---------------------------------------------------------%

\graphicspath{{figuras/prototipos/}}
\begin{figure}[H]
\centering
\includegraphics[width=1.0\textwidth]{tela-dados-cad}
\caption{Protótipo da Tela dos Dados do Equipamento Cadastrados pelo SIPAT. \textbf{Fonte: Autor.}}
\label{tela-dados-cad}
\end{figure}


%---------------------------------------------------------%

\graphicspath{{figuras/prototipos/}}
\begin{figure}[H]
\centering
\includegraphics[width=1.0\textwidth]{tela-ind-custo}
\caption{Protótipo da Tela do Indicador de Custo. \textbf{Fonte: Autor.}}
\label{tela-ind-custo}
\end{figure}

\graphicspath{{figuras/prototipos/}}
\begin{figure}[H]
\centering
\includegraphics[width=1.0\textwidth]{tela-ind-falha}
\caption{Protótipo da Tela do Indicador de Falha. \textbf{Fonte: Autor.}}
\label{tela-ind-falha}
\end{figure}

%---------------------------------------------------------%


\graphicspath{{figuras/prototipos/}}
\begin{figure}[H]
\centering
\includegraphics[width=1.0\textwidth]{tela-comp-princ}
\caption{Protótipo da Tela para Adicionar os Componentes Principais do Equipamento. \textbf{Fonte: Autor.}}
\label{tela-comp-princ}
\end{figure}

%---------------------------------------------------------%


\graphicspath{{figuras/prototipos/}}
\begin{figure}[H]
\centering
\includegraphics[width=1.0\textwidth]{tela-tipo-man}
\caption{Protótipo da Tela para Escolha do Tipo de Manutenção que será realizada no Equipamento. \textbf{Fonte: Autor.}}
\label{tela-tipo-man}
\end{figure}

\graphicspath{{figuras/prototipos/}}
\begin{figure}[H]
\centering
\includegraphics[width=1.0\textwidth]{tela-quest-man}
\caption{Protótipo da Tela do Questionário para o sistema sugerir o tipo de manutenção mais adequado para um lote de equipamentos. \textbf{Fonte: Autor.}}
\label{tela-quest-man}
\end{figure}

%---------------------------------------------------------%


\graphicspath{{figuras/prototipos/}}
\begin{figure}[H]
\centering
\includegraphics[width=1.0\textwidth]{tela-os-eq}
\caption{Protótipo da Tela de Listagem das Ordens de Serviço já Abertas para o Equipamento. \textbf{Fonte: Autor.}}
\label{tela-os-eq}
\end{figure}

\end{apendicesenv}
