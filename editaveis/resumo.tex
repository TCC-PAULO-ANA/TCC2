\begin{resumo}
 
Este trabalho apresenta a importância do gerenciamento da manutenção dos ativos de uma organização, tendo como enfoque os equipamentos eletrônicos, que por vezes possuem suas manutenções negligenciadas, sendo encaradas como uma fonte de despesa que não apresenta um retorno perceptível. São aqui apresentados os fundamentos da Gestão da Manutenção destes Equipamentos, trazendo os principais tipos de manutenção descritos na literatura, também são apresentados padrões e técnicas que regem a gestão de ativos (ISO 5500x e a PAS 55). 
A fundamentação obtida na literatura demonstra que uma má gestão das manutenções desperdiça recursos, trazendo gastos extras e não planejados e também gera soluções precárias e tardias que elevam a taxa de indisponibilidade dos ativos. Para sanar estes problemas a bibliografia define os principais tipos de manutenção: Corretiva, preventiva, preditiva, e proativa, bem como afirma que não existe uma solução geral para o problema, cabendo ao contexto e a natureza da tarefa em que esses equipamentos são aplicados a definição da melhor maneira de agir, sendo a solução particular para cada caso. 
Sendo assim, como objeto de estudo e validação, esta pesquisa retrata o atual modo gestão da manutenção de equipamentos eletrônicos empregados pela Diretoria de Manutenção de Equipamentos da Universidade de Brasília (DIMEQ/UNB) e desenha alguns de seus processos, para então sugerir melhores técnicas de controle, planejamento e coordenação para se ter ativos que possam gerar valor ao negócio, empregando na solução conceitos como Tempo Médio de Falha, Tempo Médio de Reparo, Tempo Médio Entre Falhas, disponibilidade, confiabilidade e custo. Assim como sugerir uma solução para informatizar esses dados construídos no trabalho.
A pesquisa define indicadores de desempenho da manutenção, para o caso particular da UnB, que buscam mensurar o momento de maior probabilidade de falha para os equipamentos eletrônicos e quantificar os custos das manutenções. Deste modo auxiliando o gestor a conhecer qual o melhor momento para se realizar as manutenções ou revelam dados comparativos entre os custos das manutenções ou a substituição dos equipamentos. Todavia, estes indicadores levam em consideração os tipos de equipamentos os agrupando em categorias, pois há equipamentos que o tipo de manutenção recomendada é a corretiva, e em outros pode ser a preventiva e etc.


 \vspace{\onelineskip}
    
 \noindent
 \textbf{Palavras-chaves}: manutenção, gestão de ativos, equipamentos eletrônicos, UnB, gestão da manutenção.
\end{resumo}
