\chapter{Computerized Maintenance Management System - CMMS}
\label{cmms}

Esse capítulo mostra a definição de um software CMMS, como sua utilização pode alavancar a eficácia do departamento de manutenção, e quais benefícios tantos gerenciais, quanto nos gastos relativos a manutenção, essa ferramenta pode trazer. Apesar da UnB já utilizar um sistema incorpora algumas características de um CMMS, a ferramenta pode ser considerada incompleta se comparada com as existentes no mercado.



\section{Definição}

Computerized Maintenance Management System ou em português Sistema Informatizado de Gestão de Manutenção, os CMMS, também conhecidos como Enterprise Asset Management (EAM) são softwares desenhados para simplificar a gestão da manutenção.

Para melhor entender o que são essas ferramentas a MicroMain corporation, quebrou o termo para explicar o uso de cada componente individualmente \cite{micromain}.

\textbf{Informatizado}

Significa que com a ferramenta CMMS os dados sobre a manutenção estarão guardados em um computador. Hoje em dia esse conceito é muito comum, guardar dados no computador, mas antigamente, antes dos anos 1980, esses dados eram armazenados com o uso de papel e lápis, o que tornava a manutenção muito mais reativa do que proativa, ou seja, ela era realizada quando algo de errado acontecia. 

Nessa época era um pouco difícil pensar realizar manutenção preventiva, por existir a dificuldade de manter o tracking (rastro) de qual ativo precisaria de manutenção rotineira, quando todo o histórico deles era mantido em um sistema de arquivamento físico, em uma gaveta. 

Foi então que no final dos anos 80 e começo dos anos 90, começou a surgir os CMMS, como solução para esses problemas, as organizações migraram seus dados sobre manutenção para seus computadores e começaram a ter a possibilidade de rastrear as ordens de serviço, gerar relatórios mais rápido e determinar quais ativos precisariam de manutenção preventiva, ajudando a diminuir gastos e aumentar os lucros. 

\textbf{Manutenção}

Manutenção é a atividade realizada pelos usuários do CMMS todos os dias, seja por demanda, ordem de serviço ou inspeções de rotina. O software não substitui o trabalho humano, mas ele ajuda a determinar quais tarefas estão priorizadas adequadamente e que todo está em seu devido lugar (inventário, trabalho, etc.). O software permite que o gestor e sua equipe foquem mais no trabalho e menos na parte burocrática. 

\textbf{Gestão}

É a principal função do software, que foi desenhado para que os usuários possam saber sobre o estado da manutenção, com horários, ordens de serviço, previsões de inventário e acesso a relatórios de forma imediata. 

\textbf{Sistema}

Sistema pode ser visto como a combinação geral de características do CMMS, que variam de acordo com os diferentes tipos de CMMS.

Quando explicado os componentes um à um, percebe-se qual a real função desse software, desenhado para facilitar e melhorar a forma como é realizada a Gestão da Manutenção. 

%----------------------------------------------------------------------------------------------------------------%
\section{Benefícios de se utilizar uma ferramenta CMMS}

Em empresas e organizações privadas e públicas, o setor de manutenção é tipicamente visto como um centro de gastos. Quando é necessário realizar economias, esse é um dos primeiros setores onde cortes são feitos. Mas a verdade é que o setor da manutenção age em todos os outros, assegurando que as atividades corram de forma adequada. Por isso, o uso de uma ferramenta que seja adequada, ajuda as atividades de manutenção fluírem de forma mais inteligente e eficaz.

No white paper da Q Ware \cite{qware}, uma empresa que desenvolve soluções de CMMS, são listados os benefícios tangíveis e intangíveis trazidos com o uso de um software CMMS.

\textbf{Benefícios Tangíveis}

\begin{enumerate}
	\item Menos quebra de equipamentos e de ativos.
	\item Menos custos de trabalho.
	\item Menos material e uso de peças.
	\item Redução de estoque.
	\item Economia de papel.
	\item Economia em contas de energia.
	\item Estender o ciclo de vida dos equipamentos.
	\item Medição de desempenho precisa.
\end{enumerate}

\textbf{Benefícios Intangíveis}

\begin{enumerate}
	\item Cria processos padronizados.
	\item Maior satisfação do cliente.
	\item Ambiente de trabalho menos estressante.
	\item Organiza sua carga de trabalho
	\item Facilidade de cumprir os regulamentos.
	\item Melhor comunicação.
	\item Fazer decisões melhores.
\end{enumerate}


%----------------------------------------------------------------------------------------------------------------%
%\section{O Processo de Gerenciamento de Manutenção}



%----------------------------------------------------------------------------------------------------------------%

\section{Comparativo entre CMMS Disponíveis no Mercado}

A ferramenta Capterra é um site que permitir buscar softwares voltados para diferentes tipos de negócio \cite{capterra}. Nela o usuário pesquisa sobre o software desejado e é devolvido uma série de resultados, que podem ser filtrados de acordo com a necessidade do usuário.

Por meio desta ferramenta foram encontrados diversos tipos de softwares CMMS, podendo consultar um tipo específico, por meio dos filtro, e também ver o perfil do software escolhido que traz informações sobre o fornecedor e um checklist com as features do software, uma forma prática e simples de comparação de características existentes entre os diferentes tipos de CMMS disponíveis. A partir do checklist presente na página de cada software, foi criada a Tabela~\ref{comparativo}, que possui uma visão geral das características de um apanhado de CMMS pesquisados.

Esse comparativo serviu como referência para avaliar o software atualmente utilizada na UnB, o SIPAT. Se suas funcionalidades estão alinhadas com o que existe no mercado ou se está aquém do que outras empresas e organizações utilizam. E também foi utilizado no levantamento de requisitos para o software proposto no trabalho.

\begin{landscape}
\begin{table}[H]
\centering
\caption{Comparativo entre características dos softwares CMMS. Fonte: Autor.}
\label{comparativo}
\begin{tabular}{ | p{6cm} | p{3cm} | p{2cm} | p{2cm} | p{3cm} | p{3cm} | p{2cm} | }
\hline
	      & ManWinWin & Rosmiman IWMS & comma CMMS & INTERAL Maintenance & Maintenance Assistant & Asset Bug \\ \hline
	Acompanhamento de Ativos & X & X & X & X & X & X \\ \hline
	Controle de Inventário & X & X & X & X & X & X \\ \hline
	Acesso Mobile & X & X & X & X & X & X \\ \hline
	Calendário de Planejamento & X & X & X & X & X & X \\ \hline
	Manutenção Preventiva & X & X & X & X & X & X \\ \hline
	Agendamento & X & X & X & X & X & X \\ \hline
	Rastreamento de Serviço & X & X & X & X & X & X \\ \hline
	Gestão Técnica &  & X & X & X & X & X \\ \hline
	Gestão de Ordem de Trabalho & X & X & X & X & X & X \\ \hline
\end{tabular}
\end{table}
\end{landscape}

