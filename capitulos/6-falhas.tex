\chapter{Medidas Básicas de Tolerância a Falhas}
\label{falhas}

Koren e Krishna \cite{koren2007} dizem em seu livro \emph{Fault-Tolerant System} que todo tipo de tolerância a falhas é um exercício de explorar e gerenciar redundância. Redundância é a propriedade de ter mais de um recurso do que o minimamente necessário para o trabalho sendo feito. Quando acontece uma falha, a redundância é explorada par mascarar ou contornar essas falhas, mantendo o nível desejado de funcionalidade. Eles trazem quatro tipos de redundância: de hardware, de software, de informação e de tempo. Abaixo será relatado como os autores tratam os conceitos relacionados a \textbf{tolerância a falhas}. 

A tolerância a falhas tem como objetivo tornar máquinas mais confiáveis, por isso ela tem medidas próprias. Medida é uma abstração matemática, que expressa uma observação da performance de um objeto. O truque em definir medidas adequadas é manter o subsistema largo o suficiente para que o comportamento de interesse do usuário seja capturado, e ainda assim não tão largo, para que a medida não perca o foco. 

Essas métricas medem atributos bem básicos do sistema. Duas dessas medidas são confiabilidade e disponibilidade. A definição convencional de confiabilidade \textbf{R(t)}, é a probabilidade de que o sistema esteja funcionando continuamente em um intervalo de tempo \textbf{[0,t]}.
Perto de confiabilidade está \textbf{Tempo Médio de Falhas (TMF)},  e \textbf{Tempo Médio Entre Falhas (TMEF)}. O primeiro é o tempo médio que o sistema opera até acontecer uma falha, enquanto o segundo é o tempo médio entre duas falhas consecutivas. A diferença entre as duas  é o tempo necessário para reparar o sistema depois da primeira falha. Estipulando \textbf{Tempo Médio de Reparo (TMR)}:

\begin{equation}
\label{eqn01}
	\mathbf{TMEF} = \mathbf{TMF} + \mathbf{TMR} 
\end{equation}

Disponibilidade, representado por \textbf{A(t)}, é a fração média de tempo sobre o intervalo \textbf{[0,t]} que o sistema está funcionando. Essa medida é apropriada para aplicações em que performance contínua não é vital mas onde seria caro ter o sistema fora por um período significativo de tempo. 
Disponibilidade pode ser interpretada como a probabilidade de que o sistema estará operante em um período de tempo aleatório, e só é significativo em sistemas que incluem reparo de componentes com defeito. Pode ser calculado por:

\begin{equation}
\label{eqn02}
	\mathbf{A} = \mathbf{\frac{TMF}{TMEF}} = \mathbf{\frac{TMF}{TMF + TMR}}
\end{equation}
Uma medida relacionada é Ap(t), que é a probabilidade do sistema operar em um determinado instante \textbf{t}.

É possível que um sistema de baixa confiabilidade tenha uma alta disponibilidade, por exemplo, um sistema que cai toda hora, mas se recupera em um segundo.


\section{Modelagem da falha de equipamentos eletrônicos}

Embora a maioria das causas que levam os equipamentos ao desgaste estejam ligadas a seu uso, eventualmente, podem ter ocorrido problemas durante o seu processo de fabricação, ocasionando a produção de componentes fora da especificação e mais frágeis. Estes equipamentos quais provavelmente irão apresentar problemas prematuros, quando colocados em operação. De acordo com Sanches (2010, p. 19) os equipamentos que apresentam defeitos/falhas provenientes dos processos de fabricação apresentam problemas em um período inicial, que em média corresponde a 5\% da sua vida útil – período conhecido como falhas prematuras ou precoces.

Ou quando se deseja melhorar as suas condições de uso, algumas delas estão relacionadas à escolha do equipamento mais indicado, ou quais modificações serão necessárias a serem feitas na infra-estrutura para adequadamente utilizá-lo.

Restabelecer características construtivas originais dos equipamentos, garantindo a eles um nível de desempenho esperado através da manutenção. 
Estas ações de correição e prevenção quando adotadas aumentam a eficiência operacional e o tempo de vida útil dos ativos da organização.

