\chapter{Gestão de Ativos}
\label{cap-ativos}

\section{Ativos}

\textbf{Definição:} \emph{Dentro da contabilidade, ativo é um recurso econômico. Pode ser considerado qualquer coisa tangível ou intangível. Expressa bens, valores créditos, direitos e assemelhados, que podem gerar valor econômico. Formam o patrimônio de uma pessoa sigular ou coletiva e são avaliados pelo seu custo.} \cite{sullivan2003}\cite{fulgencio2007} 

\textbf{Definição:} \emph{Ativo físico é algo que tem valor real ou potencial para uma organização.
Exemplos: plantas, instalações, equipamentos, estoques, ferramentas, materiais, edifícios, veículos etc.} \cite{nicolay2015}

Ativos são recursos que dão suporte as atividades realizadas por uma organização. E pelas definições acima constata-se que são intrísicos ao seu custo e também geram valor para o negócio. A troca constante desses ativos, por obsolescência, quebra ou falhas, podem gerar custos altos e despesas extras. Por isso é necessário gerí-los de forma adequada, de forma a preservá-los, prolongar seu tempo de vida e ter uma previsão dos gastos necessários para mantê-los. 

Controlar seu ciclo de vida pode ser uma forma de prever falhas, evitá-las e assim melhorar seu desempenho e prolongar seu tempo de uso. A gestão de ativos é uma área que vem sendo valorizada exatamente porque a indústria percebeu a necessidade de diminuir os custos com reparos e substituições.

Marcio Nicolay \cite{nicolay2015} defini vida do ativo e ciclo de vida como:

\textbf{Vida do ativo:} é o período compreendido desde sua criação até o final de sua vida. Sua vida não necessariamente termina depois do descarte.

\textbf{Ciclo de vida:} são todas as etapas envolvidas na gestão de um ativo. Quantidade de etapas e duração variam de acordo com a necessidade da organização.

