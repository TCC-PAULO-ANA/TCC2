\textbf{PARTE ANTIGA}


A European Federation of National Maintenance Societies (EFNMS) define manutenção como a combinação de toda ação técnica, administrativa e gerencial que, durante o ciclo de vida de um ativo, tenha a intenção de mantê-lo ou restaurá-lo para o estado útil, realizando a função requerida. Diz, ainda, que a manutenção tem importância máxima para o negócio e o comércio, bem como para o meio ambiente e, em geral, para a saúde e a segurança \cite{efnms}.

\section{Tipos de Manutenção}

Existem diferentes tipos de manutenção e eles são distintos e dependentes do contexto, da natureza da tarefa em que serão aplicados. Araújo \cite{araujo2015} traz a definição de quatro tipos de manutenção, sendo elas:

\begin{enumerate}
	\item \textbf{Manutenção Corretiva:} é a intervenção quando ocorre uma falha, sendo que essa falha torna o equipamento indisponível ou com baixa confiabilidade, isso não significa que ela seja uma manutenção de emergência, pois existem duas classes de manutenção corretiva.
		\begin{enumerate}
			\item \textbf{manutenção corretiva não planejada:} nela a intervenção é imediata, sem tempo para preparação do serviço.
			\item \textbf{manutenção corretiva planejada:} nela a intervenção no equipamento é programada. Pode-se optar pela operação do equipamento até ele quebrar, depende da decisão gerencial.
		\end{enumerate}
	\item \textbf{Manutenção Preventiva:} nesse tipo realiza-se a manutenção como objetivo de reduzir ou evitar a falha ou queda de desempenho do equipamento, seguindo um plano previamente elaborado, e são realizadas periodicamente. Ela procura evitar a ocorrência de falhas, por meio do conhecimento prévio de ações que devem ser tomadas.
	\item \textbf{Manutenção Preditiva:} consiste na modificação de um parâmetro de condição ou desempenho, seu acompanhamento segue uma sistemática. Utiliza instrumentos de manutenção, para prevenir falhas em equipamentos ou sistemas por meio de parâmetros diversos, que irão permitir a operação contínua do equipamento pelo máximo tempo possível.
	\item \textbf{Manutenção proativa:} tem como base a frequência na ocorrência da falha. É feito um histórico dessas ocorrências no equipamento e retiradas informações para saber qual a causa básica de falhas frequentes. Ela gera ações que está relacionadas a causa raíz da falha, com o objetivo de aumentar o tempo de vida do equipamento.
\end{enumerate}

Esses quatro tipos mostram conceitos diferentes, mas também muito próximos, trazendo uma certa dificuldade em pensar qual seria melhor para um determinado contexto a ser trabalhado. Muitas vezes eles podem se tornar complementares. Garrido \cite{garrido} ainda diz que é muito difícil encontrar uma aplicação para cada conceito, e que por isso, muitas vezes é mais valoroso usar mais de um tipo ao contexto sendo trabalhado. Por isso ele traz a ideia de Modelos de Manutenção, que misturam os tipos acima citados e assim podem atender as necessidades de manutenção de um certo tipo de equipamento.

\section{Modelos de Manutenção}
\label{sec_modelos_manutencao}

A seguir serão explicados os modelos de manutenção definidos por Garrido.

\begin{enumerate}
	\item \textbf{Modelo Corretivo:} consiste no modelo mais básico, sendo aplicado em equipamentos com menor criticidade no seu uso, ou seja, seu uso não compromete fatalmente a tarefa à qual ele dá suporte. Os defeitos encontrados não são um problema econômico ou técnico. 
	\item \textbf{Modelo Condicional:} utiliza as atividades do modelo corretivo, acrescentando uma série de testes que irão definir uma ação seguinte. Caso encontre-se algo errado nos testes realizados, é agendada uma intervenção. Esse modelo é aplicável em equipamentos pouco utilizados ou cujo a probabilidade de falha é pequena.
	\item \textbf{Modelo Sistemático:} consiste em executar um conjunto de atividades, medições e experimentos independente do estado equipamento, para então reparar falhas que poderão surgir. É utilizado em equipamentos de média disponibilidade e nível de importância no sistema, onde sua falha possa trazer algum tipo de problema. Um exemplo de uso seria um reator descontínuo. 
	\item \textbf{Modelo de Manutenção de Alta Disponibilidade:} é o modelo mais exaustivo, por ser aplicado em equipamentos que necessitem de uma disponibilidade acima de 90\%, esse tipo de equipamento gera altos custos na produção caso ocorra uma falha nele, pois caso ele tenha que ficar parado, pode comprometer as atividades que ele executa. Por isso não existe tempo para parar o equipamento, como exige os outros modelos citados. Assim são utilizadas técnicas da manutenção preditiva, para conhecer o estado do equipamento e também realizar manutenções programadas, com revisões completas, com um frequência anual ou superior, que tem o intuito de substituir peças sujeitas a desgastes ou com maior probabilidade de falhas. 
	\\
	Este modelo também traz a necessidade de se realizar manutenções corretivas e reparações provisórias, que manterão o equipamento funcionando até a próxima revisão. Exemplos de uso deste modelo: Fornos de alta temperatura e depósitos de reatores.
\end{enumerate} 

Além de modelos que podem ser criados por cada organização, de acordo com a necessidade e o tipo de equipamentos existentes, também pode se criar classificações para os tipos de equipamentos, sistemas ou componentes. No exemplo utilizado por \cite{valeria2013}, foram utilizados três tipos de classificações para esses itens, sendo elas:

\begin{itemize}
	\item críticos: podem parar as atividades ou causar danos;
	\item não críticos: apresentam histórico de alto custo operacional, ou de reparo, ou substituição, podem demandar longo tempo para aquisição ou são considerados obsoletos;
	\item podem funcionar até falhar.
\end{itemize}

Classificar os elementos ou criar modelos de manutenção, são formas de se facilitar o controle sobre os ativos e saber suas características e o papel que exercem no negócio.

\subsection{Origem dos danos}

É fato que todos componentes das máquinas, equipamentos e instalações, bem como os prédios e as benfeitorias inevitavelmente se desgastem ao longo tempo, necessitando assim de um conjunto de tratativas e cuidados técnic

os periódicos para se manterem em condições de pleno funcionamento, devido a isso a manutenção, hoje vista como área estratégica, assume uma importância fundamental da gestão dos ativos, com reflexos diretos nos nível de operação, logística, Disponibilidadeidade e confiabilidade dos meios produtivos. Consoante ao escopo deste trabalho entenda-se aqui ativos como os equipamentos eletroeletrônicos, todavia muitos dos princípios e conhecimentos aqui abordados são úteis a gestão eficiente de todos os demais ativos de uma organização. 

São múltiplas as causas que levam a deterioração e ao desgaste dos ativos de uma organização, podendo ter natureza mecânica, elétrica, térmica, química ou operacional, sendo na maioria das vezes uma associação destas. Segundo Seeling \cite{seeling2000} as causas corriqueiras que merecem destaque são:

\begin{itemize}
	\item o atrito entre peças móveis em operação;
	\item os esforços realizados pelos componentes em funcionamento normal;
	\item os esforços estáticos e dinâmicos suportados pelas estruturas;
	\item o calor, frio, umidade;
	\item a pressão, vibração, oxidação (ferrugem);
	\item a sujeira acumulada, abrasão;
	\item a corrosão pelo ataque químico dos elementos existentes no ambiente;
	\item as intempéries que agem sobre os materiais expostos ao tempo.
\end{itemize}


A consequência cumulativa das fontes de desgaste, descritas acima, são os defeitos e as falhas. Sendo o defeito definido como a situação que não impede o funcionamento do equipamento, todavia pode acarretar a curto ou longo prazo a sua indisponibilidade, já a falha é descrita como as ocorrências que impedem o equipamento ou instalação de funcionar tornando-o indisponível, ou seja, é quando deixamos de executar certas tarefas, porque no momento em que o equipamento foi solicitado, ele falhou. Por esta razão, e tendo em vista que as falhas nos equipamentos podem representar grandes perdas para a imagem das organizações, a redução e o controle destas fontes de desgaste e deterioração são essenciais.

As raízes das falhas de um equipamento estão nos danos sofridos pelas suas peças ou componentes, pois normalmente um equipamento não quebra totalmente de uma vez, mas para de funcionar quando alguma parte primária de seu conjunto se danifica. A parte primária avariada pode estar dentro do hardware, em um componente ou ao longo de um circuito em diversos pontos, por exemplo, a fonte é uma parte vital para que um circuito funcione, assim como suas trilhas para a passagem de corrente elétrica.

Sendo assim, antes de pensar em manutenção e seus indicadores de performace que tanto podem contribuir para a melhoria da gestão dos ativos, as empresas e organizações públicas precisam conhecer em quais condições irão fazer uso de seus equipamentos, e quais são suas condições ideais de operação, pois precisam adotar ações adequadas e com o menor custo. Esta compreensão a auxiliará na tomada de decisão em dois momentos, no momento da aquisição e no do uso produtivo desses equipamentos, pois não é inteligente, nem economicamente atrativo fazer uso de equipamentos em condições diferentes das quais foi planejado, pois por muitas vezes pode ser observado, para equipamentos eletrônicos, que o custo médio de manutenção, ao longo de sua vida útil pode ser muito superior ao custo de aquisição do próprio equipamento. Entenda-se tempo de vida útil a fase de utilização do ciclo de vida do ativo, ou seja, é o período de tempo iniciado no momento de sua aquisição (entrada em operação), com duração estimada de tempo (meses ou anos) que possa cumprir corretamente a função técnica para o qual foi concebido, durante o qual o mesmo realiza um trabalho com rentabilidade. 

Muitas são as considerações a serem feitas no momento da escolha de um equipamento eletroeletrônico, sendo algumas delas a disponibilidade de peças de reposição, qualidade, confiabilidade, manutenabilidade, dimensões físicas, infra-estrutura necessária, prazo de entrega, preço e treinamentos necessários para instalação, operação e manutenção. 

Do ponto de vista da organização, a melhor escolha é a que apresenta o menor custo de ciclo de vida (utilização) do ativo, e o maior tempo de operação sem manutenção. Isto é, os menores custos envolvidos na aquisição de equipamentos para testes, manutenção e partes de reposição (inclusive um equipamento completo), bem como o custo envolvido com a parada do equipamento no período da manutenção.


\begin{table}[]
\centering
\caption{Questionário repondido na entrevista com o funcionário da Dimeq. Fonte: Autor.}
\label{questionario}
\begin{tabular}{ |p{8cm}| p{8cm} | p{3cm} |}
\hline
\multicolumn{2}{|c|}{\textbf{Questionário Respondido na Entrevista}} \\ \hline
Perguntas & Respostas \\ \hline
\textbf{\begin{tabular}[c]{@{}l@{}}Como são inventariados os equipamentos\\  da Universidade de Brasília?\end{tabular}} & Pelo SIPAT. \\ \hline
\textbf{Como é feito o processo de manutenção?} & \begin{tabular}[c]{@{}l@{}}Para iniciar o pedido de manutenção o usuário entra no SIPAT, \\ descreve o problema, insere o tipo (reparo), depois o técnico \\ vai ao local, se for possível ele realiza o reparo, senão, o \\ equipamento é recolhido. Em seguida o equipamento é \\ devolvido ao usuário.\end{tabular} \\ \hline
\textbf{\begin{tabular}[c]{@{}l@{}}Quais são as áreas envolvidas no processo de\\ manutenção?\end{tabular}} & Dimeq e prefeitura. \\ \hline
\textbf{\begin{tabular}[c]{@{}l@{}}Qual é o procedimento para se solicitar a manutenção\\ de um equipamento?\end{tabular}} & A manutenção é solicitada,pelo SIPAT. \\ \hline
\textbf{\begin{tabular}[c]{@{}l@{}}Este setor emprega algum método para saber a \\ frequência de quebra de cada equipamento ou \\ modelo de equipamento? Se existir, como é feito?\end{tabular}} & \begin{tabular}[c]{@{}l@{}}Pelo patrimônio (SIPAT) é possível saber \\ quantas vezes o equipamento foi para a \\ manutenção.\end{tabular} \\ \hline
\textbf{É conhecido o valor gasto em cada manutenção prestada?} & Não. \\ \hline
\textbf{É realizado algum tipo de manutenção preventiva?} & Sim, existe uma seção de,manutenção preventiva. \\ \hline
\textbf{\begin{tabular}[c]{@{}l@{}}Quem realiza o serviço de manutenção de \\ equipamentos na Universidade? (Equipe interna? \\ Algum prestador de serviço? A própria assistência \\ técnica do fabricante?)\end{tabular}} & \begin{tabular}[c]{@{}l@{}}Se o equipamento estiver na garantia, \\ a empresa fornecedora. Sem garantia, \\ uma prestadora de serviço.\end{tabular} \\ \hline
\textbf{\begin{tabular}[c]{@{}l@{}}Se a manutenção for interna, como são adquiridas \\ as peças de reposição? São originais? \\ Possuem garantia?\end{tabular}} & \begin{tabular}[c]{@{}l@{}}É feita uma lista de peças de reposição, \\ contendo quais faltam, as mais usadas. \\ Peças muito caras a UnB não libera recurso,\\ usuário geralmente compra. \\ O manual vai direto para o usuário.\end{tabular} \\ \hline
\textbf{\begin{tabular}[c]{@{}l@{}}É dado algum prazo de garantia aos \\ equipamentos que passaram por manutenção ? \\ Se sim, qual?\end{tabular}} & Sim, 10 dias úteis. \\ \hline
\end{tabular}
\end{table}


------------------------------------------------------------------------------------------------------------

\chapter{Resultados esperados}

Tendo em vista que este trabalho é apenas uma entrega parcial, apresentada na disciplina de Trabalho de Conclusão de Curso 1, a título de controle de seu desenvolvimento e aceitação da proposta de trabalho. Temos como expectativa para o momento da entrega final, conhecer com propriedade os modos de operacionalização desenvolvidos pela Diretoria de Manutenção de Equipamentos da Universidade de Brasília – DIMEQ/UNB tendo seus pontos suscetíveis de melhoria mapeados.

Esperamos ainda construir indicadores úteis para medir o desempenho da manutenção de Equipamentos eletrônicos para o caso específico da Universidade de Brasília, com base na análise do ciclo de vida útil destes equipamentos usando para tal um software iterativo de alto desempenho voltado para o cálculo numérico (MATLAB).

A partir destes indicadores iremos desenhar uma ferramenta computadorizada para armazenar e tratar tais informações, com vistas a qualificar as decisões relacionadas a manutenções, do ponto de vista do custo e da viabilidade da manutenção. Esta ferramenta ainda trará aos gestores importantes informações que auxiliem a DIMEQ a implantar políticas de manutenções preventivas eficientes. Como estudo de caso, este trabalho avaliará a recente decisão adotada pela Faculdade do Gama, de trocar seus projetores multimídias por televisores.

Como reflexos positivos indiretos, esperamos estar contribuindo para com a Universidade fazendo com que ela economize recursos, bem como melhore os índices de disponibilidade de equipamentos, aumentando sua produção científicas e as atividades de apoio a docência e ensino.


------------------------------------------------



\subsection{Modelo de Domínio}

O Modelo de Domínio é utilizado para representar classes conceituais ou objetos do mundo real em um domínio, ou seja, mostra de forma visual qual o problema a ser resiolvido pelo sistema. Para isso, é necessário identificar os conceitos relacionados aos requisitos, analisar o problema que o sistema buscar resolver de forma conceitual.

Ele não pertence ao domínio da solução, como o diagrama de classe, e portanto não deve ser utilizado para modelar a arquitetura do software, seu papel é representar o problema por meio de um diagrama. Não deve ser associado

Sua modelagem é feita utilizando os elementos da notação UML para diagrama de classes. É composto basicamente por:

\begin{itemize}
	\item Conceitos (classes conceituais);
	\item Atributos;
	\item Relacionamento entre classes conceituais.
\end{itemize}



Agradecemos primeiramente à Deus por sempre estar ao nosso lado nessa caminhada. Ao nosso orientador Prof. Dr. Edgard Costa e ao Prof. Dr. Leonardo Aguayo, nosso coorientador. E as nossas famílias pelo apoio incondicional. 





