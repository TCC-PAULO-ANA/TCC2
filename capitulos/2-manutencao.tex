\chapter{Manutenção}
\label{cap-manutencao}

\section{Conceito}

\section{Tipos de Manutenção}

Existem diferentes tipos de manutenção e eles são distintos e dependentes do contexto, da natureza da tarefa em que serão aplicados. Araújo \cite{araujo2015} traz a definição de quatro tipos de manutenção, sendo elas:

\begin{enumerate}
	\item \textbf{Manutenção Corretiva:} é a intervenção quando ocorre uma falha, sendo que essa falha torna o equipamento indisponível ou com baixa confiabilidade, isso não significa que ela seja uma manutenção de emergência, pois existem duas classes de manutenção corretiva.
		\begin{enumerate}
			\item \textbf{manutenção corretiva não planejada:} nela a intervenção é imediata, sem tempo para preparação do serviço.
			\item \textbf{manutenção corretiva planejada:} nela a intervenção no equipamento é programada. Pode-se optar pela operação do equipamento até ele quebrar, depende da decisão gerencial.
		\end{enumerate}
	\item \textbf{Manutenção Preventiva:} nesse tipo realiza-se a manutenção como objetivo de reduzir ou evitar a falha ou queda de desempenho do equipamento, seguindo um plano previamente elaborado, e são realizadas periodicamente. Ela procura evitar a ocorrência de falhas, por meio do conhecimento prévio de ações que devem ser tomadas.
	\item \textbf{Manutenção Preditiva:} consiste na modificação de um parâmetro de condição ou desempenho, seu acompanhamento segue uma sistemática. Utiliza instrumentos de manutenção, para prevenir falhas em equipamentos ou sistemas por meio de parâmetros diversos, que irão permitir a operação contínua do equipamento pelo máximo tempo possível.
	\item \textbf{Manutenção proativa:} tem como base a frequência na ocorrência da falha. É feito um histórico dessas ocorrências no equipamento e retiradas informações para saber qual a causa básica de falhas frequentes. Ela gera ações que está relacionadas a causa raíz da falha, com o objetivo de aumentar o tempo de vida do equipamento.
\end{enumerate}

Esses quatro tipos mostram conceitos diferentes, mas também muito próximos, trazendo uma certa dificuldade em pensar qual seria melhor para um determinado contexto a ser trabalhado. Muitas vezes eles podem se tornar complementares. Garrido \cite{garrido} ainda diz que é muito difícil encontrar uma aplicação para cada conceito, e que por isso muitas vezes é mais valoroso usar mais de um tipo ao contexto sendo trabalhado. Por isso ele traz a ideia de Modelos de Manutenção, que misturam os tipos acima citados e assim podem atender as necessidades de manutenção de um certo tipo de equipamento.

\section{Modelos de Manutenção}

A seguir serão explicados os modelos de manutenção definidos por Garrido.

\begin{enumerate}
	\item \textbf{Modelo Corretivo:} consiste no modelo mais básico, sendo aplicado em equipamentos com menor criticidade no seu uso, ou seja, seu uso não compromete fatalmente a tarefa à qual ele dá suporte. Os defeitos encontrados não são um problema econômico ou técnico. 
	\item \textbf{Modelo Condicional:} utiliza as atividades do modelo corretivo, acrescentando uma série de testes que irão definir uma ação seguinte. Caso encontre-se algo errado nos testes realizados, é agendada uma intervenção. Esse modelo é aplicável em equipamentos pouco utilizados ou cujo a probabilidade de falha é pequena.
	\item \textbf{Modelo Sistemático:} consiste em executar um conjunto de atividades, medições e experimentos independente do estado equipamento, para então reparar falhas que poderão surgir. É utilizado em equipamentos de média disponibilidade e nível de importância no sistema, onde sua falha possa trazer algum tipo de problema. Um exemplo de uso seria um reator descontínuo. 
	\item \textbf{Modelo de Manutenção de Alta Disponibilidade:} é o modelo mais exaustivo, por ser aplicado em equipamentos que necessitem de uma disponibilidade acima de 90\%, esse tipo de equipamento gera altos custos na produção caso ocorra uma falha nele, pois caso ele tenha que ficar parado, pode comprometer as atividades que ele executa. Por isso não existe tempo para parar o equipamento, como exige os outros modelos citados. Assim são utilizadas técnicas da manutenção preditiva, para conhecer o estado do equipamento e também realizar manutenções programadas, com revisões completas, com um frequência anual ou superior, que tem o intuito de substituir peças sujeitas a desgastes ou com maior probabilidade de falhas. 
	\\
	Este modelo também traz a necessidade de se realizar manutenções corretivas e reparações provisórias, que manterão o equipamento funcionando até a próxima revisão. Exemplos de uso deste modelo: Fonos de alta temperatura e depósitos de reatores.
\end{enumerate} 


\section{Normas}

\begin{flushright}
	“\textit{A normalização é tecnologia consolidada, que nos
permite confiar e reproduzir infinitas vezes determinado
procedimento, seja na área industrial, seja no campo de
serviços, ou em programas de gestão, com mínimas
possibilidades de errar...
\\
...
\\
Elaborar uma norma técnica é compartilhar
conhecimento, promover a competitividade, projetar a
excelência e suas melhores consequências nos planos
econômico, social e ambiental.}”
\\
Pedro Buzatto Costa
\\
HISTÓRIA DA NORMALIZAÇÃO BRASILEIRA
\end{flushright}

A citação acima diz em poucas palavras a importância de se ter normas, padrões e especificações que possam assegurar a qualidade em diferentes áreas de conhecimento.

Na área da manutenção e gestão de ativos não é diferente. A preocupação com ativos físicos tem se tornado cada vez mais importantes, pois observou-se que garantir a confiabilidade, desempenho, conservar e aumentar o tempo de vida deles pode trazer benefícios significantes, sendo um dos mais importantes, o custo, gastos que podem ser poupados.

\subsection{PAS 55}

Surgiu, em reposta a uma demanda da indútria, o documento PAS 55. Publicado em 2004 pela BSI e revisado em 2008, essa especificação é composta por definições claras e 28 requisitos, considerados necessários para implantar e auditar um eficiente sistema de gestão para todo o ciclo de vida de qualquer ativo físico. A PAS 55 foi dividido em duas parte, traduzidas pela ABRAMAN como:

\begin{itemize}
	\item \textbf{Parte 1:} Especificação para a gestão otimizada de ativos físicos;
	\item \textbf{Parte 2:} Diretrizes para aplicação do PAS 55-1. 
\end{itemize} 

Sua estrutura consiste em um ciclo baseado no PDCA, como mostra a figura abaixo.

\textcolor{red}{colocar figura}


A Figura 1 foi adaptada por \cite{valeria2013} e mostra a estrutura do Sistema de Gestão de Ativos. A PAS classifica 5 categorias de ativos:

\begin{enumerate}
	\item{Ativos Físicos}
	\item{Ativos Humanos}
	\item{Ativos de Informação}
	\item{Ativos Financeiros}
	\item{Ativos Intangíveis}
\end{enumerate}

Contudo seu escopo está focado nos \textbf{ativos físicos}. Os outros só são considerados caso estejam realacionados com os ativos.

Segurança, confiabilidade, disponibilidade, infraestrutura e custo são associados a atividades contidas na gestão de ativos. 
Realizar a gestão dos ativos é importante para, conhecer a confiabilidade e a disponibilidade dos sistemas e componentes críticos ao longo do tempo de operação, os riscos inerentes à operação e manutenção, as probabilidades de ocorrências de eventos não desejáveis que afetem a segurança das pessoas e do meio-ambiente.


\subsection{ISO 55000}

A importânia da PAS 55 foi reconhecida quando baseada nela, em janeiro de 2014, foi aprovada a ISO 55000, o primeiro padrão internacional para gestão de ativos. 

A ISO é uma organização internacional e não gonvernamental independente, tendo adesão de 161 organismos necionais de normalização, sendo um deles a ABNT. Seu objetivo é construir padrões internacionais relevantes para o mercado e que deêm suporte a inovação e traga soluções para desafios globais .





