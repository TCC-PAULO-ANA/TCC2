\chapter{Manutenção}
\label{cap-manutencao}

\section{Tipos}

Existem diferentes tipos de manutenção,  Garrido(citar) diz que  cinco tipos foram distinguidos, e eles são diferentes e dependentes do contexto, da natureza da tarefa em que serão aplicados. Sendo eles: 

\begin{enumarate}
	\item \textbf{Manutenção Corretiva:} conjunto de atividades destinadas a corrigir defeitos encontrados nos equipamentos. São comunicados ao departamento de manutenção pelos usuários dos equipamentos.
	\item \textbf{Manutenção Preventiva:} o objetivo é manter um certo nível de serviço no equipamento, para isso as intervenções são programadas. Os equipamentos são inspecionados sistematicamente.
	\item \textbf{Manutenção Preditiva:} informa constantemente o estado e a capacidade operacional das instalações sabendo os valores de algumas variáveis.    

%

Já Araújo araújo2015(citar) define quatro tipos de manutenção, sendo elas:
\begin{enumerate}
	\item \textbf{Manutenção Corretiva:} é a intervenção quando ocorre uma falha, sendo que essa falha torna o equipamento indisponível ou com baixa confiabilidade, isso não siginifica que ela seja uma manutenção de emergência, pois existem duas classes de manutenção corretiva.
		\begin{enumerate}
			\item \textbf{manutenção corretiva não planejada:} nela a intervenção é imediata, sem tempo para preparação do serviço.
			\item \textbf{manutenção corretiva planejada:} nela a intervenção no equipamento é programada. Pode-se optar pela operação do equipamento até ele quebrar, depende da decisão gerencial.
		\end{enumerate}
	\item \textbf{Manutenção Preventiva:} nesse tipo realiza-se a manutenção como objetivo de reduzir ou evitar a falha ou queda de desempenho do equipamento, seguindo um plano previamente elaborado, e são realizadas periodicamente. Ela procura evitar a ocorrência de falhas, por meio do conhecimento prévio de ações que devem ser tomadas.
	\item \textbf{Manutenção Preditiva:} basea-se na modificação de um parâmetro de condição ou desempenho, seu acompanhameno segue uma sistemática. Utiliza instrumentos de manutenção, para previnir falhas em equipamentos ou sistemas por meio de parâmetros diversos, que irão permitir a operação contínua do equipamento pelo máximo tempo possível.
	\item \textbf{Manutenção protiva:} tem como base a frequência na ocorrência da falha. É feito um histórico dessas ocorrências no equipamento e retiradas informações para saber qual a causa básica de falhas frequentes. Ela gera ações que está relacionadas a causa raíz da falha, com o objetivo de aumentar o tempo de vida do equipamento.
\end{enumerate}



%

