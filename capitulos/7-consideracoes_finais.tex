\chapter{Resultados esperados}

Tendo em vista que este trabalho é apenas uma entrega parcial, apresentada na disciplina de Trabalho de Conclusão de Curso 1, a título de controle de seu desenvolvimento e aceitação da proposta de trabalho. Temos como expectativa para o momento da entrega final, conhecer com propriedade os modos de operacionalização desenvolvidos pela Diretoria de Manutenção de Equipamentos da Universidade de Brasília – DIMEQ/UNB tendo seus pontos suscetíveis de melhoria mapeados.

Esperamos ainda construir indicadores úteis para medir o desempenho da manutenção de Equipamentos eletrônicos para o caso específico da Universidade de Brasília, com base na análise do ciclo de vida útil destes equipamentos usando para tal um software iterativo de alto desempenho voltado para o cálculo numérico (MATLAB).

A partir destes indicadores iremos desenhar uma ferramenta computadorizada para armazenar e tratar tais informações, com vistas a qualificar as decisões relacionadas a manutenções, do ponto de vista do custo e da viabilidade da manutenção. Esta ferramenta ainda trará aos gestores importantes informações que auxiliem a DIMEQ a implantar políticas de manutenções preventivas eficientes. Como estudo de caso, este trabalho avaliará a recente decisão adotada pela Faculdade do Gama, de trocar seus projetores multimídias por televisores.

Como reflexos positivos indiretos, esperamos estar contribuindo para com a Universidade fazendo com que ela economize recursos, bem como melhore os índices de disponibilidade de equipamentos, aumentando sua produção científicas e as atividades de apoio a docência e ensino.

\chapter{Conclusão}

Em um momento de crise econômica, como o enfrentado pela Economia Brasileira nos últimos tempos, as políticas orçamentárias ficam muito mais duras, tanto para o Governo quando para a Iniciativa Privada, tendo como conseqüência o corte do orçamento de muitos órgãos da Administração Pública e Empresas Privadas.

Sendo assim, as instituições precisam inovar em sua gestão de ativos tendo que fazer mais com menos recursos que outrora. E com a função manutenção não é diferente, ela precisa se adequar a nova realidade, tendo em conta também que seu público tende a ficar cada vez mais exigente, para tal as práticas eficiente de gestão de ativos precisam se consideradas.
Para isto é fundamental a observância do \lq\lq estado da arte\rq\rq da área pelas organizações, procurando sempre adequá-lo a realidade particular que se deseja empregar, pois uma solução geral para esta área da manutenção não existe. 

Avaliar como está o desempenho da manutenção de uma empresa, deve ser o 1º passo para a gestão eficiente desta área. Para isto é essencial que se desenvolvam indicadores de desempenho da manutenção como: Tempo Médio de Falha, Tempo Médio de Reparo, Tempo Médio Entre Falhas, disponibilidade, confiabilidade, e custo.

Porém não basta a simples obtenção desses dados, é preciso avaliá-los para que eles contribuam efetivamente nas de decisões gerenciais da Manutenção, para tal uma ferramenta computacional é indicada. Ela apresentará ao gestor dados como o melhor momento para se fazer manutenções, e se ela compensa do ponto de vista financeiro.

A avaliação dos serviços de manutenção de equipamentos eletrônicos da UnB mostra o enorme desafio que é sua gestão, tendo em vista o grande número de unidades, inúmeras variedades e especificações de equipamentos e diferentes tipos de utilização. Constatou-se ainda que a Diretoria da Manutenção de Equipamentos da Universidade adota predominantemente a manutenção corretiva, e passa a visão errônea de que o mais adequado em todos os casos é famoso quebrou-concerta, não avaliando e quantificando suas consequências.

Por tudo isso, revelou-se bastante oportuna esta pesquisa e seus resultados esperados. Tendo em vista o potencial de melhoria que poderá ser alcançado tendo reflexo na economia de recursos e na melhoria dos índices de disponibilidade, confiabilidade e segurança dos Equipamentos Eletrônicos.




