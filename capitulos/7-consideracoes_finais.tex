\chapter{Conclusão}

Inicialmente este trabalho realizou uma revisão bibliográfica acerca da função manutenção, abordando aspectos sobre sua importância, gerenciamento, metodologias e normas de operação, buscando conhecer o "estado da arte" do desenvolvimento deste tema. Após esta etapa foi avaliada a situação da gestão dos serviços de manutenção da UnB, que se mostrou um enorme desafio, tendo em vista o grande número de unidades, as inúmeras variedades e especificações de equipamentos e, ainda, o quadro reduzido de funcionários responsáveis por esta função.

Constatou-se ainda que a Diretoria da Manutenção de Equipamentos da Universidade - DIMEQ adota predominantemente a manutenção corretiva, passando a visão errônea de que o mais adequado em todos os casos é famoso quebrou-concerta, não avaliando e quantificando suas consequências. Deste modo, visando contribuir com o processo, foi sugerida uma solução que informatize e simplifique o gerenciamento da manutenção na UnB.

A solução proposta compreende os entraves burocráticos e legais inerentes ao serviço público ao não mudar completamente o sistema de gerenciamento atual da UnB - o SIPAT. O que é sugerido então, são novas funcionalidades àquele sistema aplicando conceitos e técnicas adquiridos na pesquisa que otimizam o serviço de manutenção, ao mesmo tempo que constrói indicadores de desempenho (KPIs) que auxiliam os gestores na fundamentação de suas decisões administrativas, dando maior previsibilidade e eficiência a área de manutenção.

Outras funções dos indicadores construídos pela solução são a identificação das possíveis falhas da gestão da manutenção de equipamentos da UnB, a eleição do melhor momento para se fazer manutenções e a viabilidade financeira da ação. Como ficou comprovada pela pesquisa, a etapa de escolha dos indicadores é bem particular de cada instituição. Para o estudo de caso particular da UnB, foram adotados os seguintes indicadores: Tempo Médio de Falha, Tempo Médio de Reparo, Tempo Médio entre Falhas, Disponibilidade, Confiabilidade, Satisfação do Cliente e Custo.

A modelagem da solução foi realizada por meio dos artefatos convencionais para o desenvolvimento de um software, a saber: A construção do processo AS IS, o levantamento dos requisitos, a construção do Processo TO BE e etc. Paralelamente a esta fase, foram avaliados os sistemas informatizados de gestão de manutenção (CMMS) adotados por outras organizações congêneres e outros CMMS existentes no mercado, fazendo um comparativo com a solução proposta.

Cabe aqui mencionar que foi realizada a prototipação de telas do sistema sugerido, utilizando dados reais de uso e informando todo o processo que o usuário irá seguir, obtendo-se, assim, uma visão geral do que é o sistema e qual o seu objetivo. Todo esse processo foi alinhado e ratificado por membros da equipe do DIMEQ e do CPD da universidade.

Dentro do processo de construção do software foi levantado pela teoria um algoritmo para a definição da melhor técnica de manutenção (corretiva, preventiva, preditiva etc) que deve ser aplicada a cada equipamento, onde foram levadas em conta a criticidade do aparelho para a instituição, a segurança, além das questões operacionais, ou seja, quanto uma falha no equipamento impacta em custo, qualidade e tempo de reparo. A atribuição da melhor técnica de manutenção será gerada pelo sistema de forma automática, por meio da resposta a um questionário.

A etapa seguinte desta pesquisa deve ser a implementação e implantação da solução informatizada na Universidade de Brasília, bem como a mensuração de sua eficácia na solução do problema de gestão da manutenção. Acreditamos que a DIMEQ, através desta ferramenta computacional, conseguirá alcançar uma melhor manutenibilidade e fiabilidade dos equipamentos.

Apesar de o mercado-alvo da solução proposta por este trabalho ser a UnB, seus conceitos são facilmente aplicáveis e modificáveis para o uso em outras instituições públicas ou privadas de vários segmentos, tendo em vista ser extremamente ampla o uso da função manutenção. Um dos setores que essencialmente deveria adotar tais conceitos é o da saúde pública, uma vez que a indisponibilidade de aparelhos clínicos pode comprometer o tratamento de doenças ou até provocar a perda de vidas humanas. Entretanto, o que temos observado na saúde pública brasileira é a negligência na manutenção de aparelhos hospitalares, causando a ineficiência e a baixa qualidade desta prestação de serviço público.

Por tudo isso, revelou-se bastante oportuna esta pesquisa e os resultados alcançados, tendo em vista o potencial de melhoria que poderá ser alcançado, com reflexos diretos na economia de recursos e na melhoria dos índices de disponibilidade, confiabilidade e segurança dos equipamentos da universidade. Isto poderá ensejar de uma forma direta um aumento na qualidade da produção de conhecimento e conteúdo para a comunidade acadêmica da UnB como um todo.


