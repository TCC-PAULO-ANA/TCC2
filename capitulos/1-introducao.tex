\chapter{Introdução}
\label{cap-introducao}

A gestão dos ativos (Asset Management) de uma organização, seja ela pública ou privada, é cada vez mais adotada ao redor do mundo como uma ferramenta para enfrentar os desafios econômicos impostos por um mercado globalizado e diversificado. Ainda, em relação ao Poder Público, a sociedade exige maior transparência e eficiência no emprego de seus tributos, uma vez que uma administração eficiente tem a capacidade de aperfeiçoar o desempenho técnico e econômico dos ativos, acelerar o retorno sobre os investimentos realizados e colaborar com o planejamento organizacional, trazendo previsibilidade e controle dos gastos envolvidos. 

Dentre as áreas da gestão de ativos, a manutenção é geralmente encarada como uma despesa que se deseja ao máximo adiar, não sendo uma prática suficientemente valorizada. Esta opinião é compartilhada em razão da manutenção ser uma fonte de custo que, aparentemente, não acrescenta um valor perceptível ao cliente final do produto ou serviço e, ainda, gera indisponibilidades momentâneas no uso de bens e recursos. Todavia, inevitavelmente o uso e a a ação do tempo fazem com que equipamentos e instalações se desgastem, necessitando periodicamente de reparos, regulagens e limpezas para que continuem operando eficientemente.

A efetiva gestão da manutenção tem se tornado cada vez mais necessária nas organizações, na medida em que a execução dos processos com eficiência contribuem para a otimização dos recursos e a redução dos desperdícios.  Neste sentido, a correta rotina de manutenções dos ativos e instalações se faz importante porque o custo de se realizar uma manutenção regular pode se tornar muito pequeno quando comparado ao custo de uma interrupção na produção ou atividade.

Na obra The Importance of Maintenance , Steve krar \cite{krar2009} definiu como sendo um dos principais propósitos da manutenção, garantir que todos os equipamentos estejam funcionando com 100\% de eficência o tempo todo, por meio de inspeções diárias e pequenos ajustes, os quais ajudarão na detecção de problemas menores, diminuindo a chance de esses problemas se tornarem maiores ou até incorrigíveis. Krar diz ainda que para se atingir uma manutenção regular e eficaz, é preciso que haja participação de todos, desde o alto executivo até as pessoas do operacional.

Neste contexto se insere o objeto de estudo deste trabalho: A manutenção de equipamentos eletrônicos da Universidade de Brasília – UnB. Esta universidade detém diversos laboratórios de pesquisa e certificação de alto valor e tecnologia, com grande número de aparelhos e equipamentos científicos de sistemas complexos. todavia como outras instituições públicas de ensino superior, adotou políticas para a aquisição de um parque cientifico e tecnológico sem pensar em aspectos importantes para a manutenção, o que atrapalha os procedimentos em caso de defeitos, cometendo erro de não verificar a existência de meios humanos e materiais para a manutenção dos equipamentos, além disso, dispõe de processos de gestão de manutenções pouco informatizados, não possuindo indicadores de desempenhos claros que auxiliem seus gestores a tomada de decisão. Segundo \cite{limacastilho2006} isso reflete, no caso particular da Universidade de Brasília, numa taxa de indisponibilidade exagerada dos equipamentos e instalações importantes para os laboratórios de ensino, pesquisa e de apoio administrativo, tendo como conseqüência a diminuição da capacidade produtiva da instituição e a insatisfação daqueles que dependem desse serviço.

Assim, a confiança nos serviços oferecidos pela Diretoria de Manutenção de Equipamentos – DIMEQ, unidade responsável por prover a manutenção e o reparo de equipamentos da Universidade, fica comprometida. 

Portanto, este trabalho tem como escopo fornecer fundamentos e indicadores de desempenho que auxiliem na gestão da manutenção dos equipamentos da Universidade, mais precisamente dos equipamentos eletrônicos. Estes são de maneira geral aqueles que possuem na maioria dos seus circuitos, componentes eletrônicos, tais como: Equipamentos biomédicos e de análise clínica, de laboratório, de som e de imagem, e aqueles que promovem a infra-estrutura de rede e comunicação. Cita-se como exemplo: Os monitores de vídeo para microcomputadores, equipamentos de som e projetores de multimídia, balanças eletrônicas, suítes, roteadores e outros.

Entende-se por gestão \lq\lq gerenciar é previnir e planejar, comandar, coordenar e controlar \rq\rq, Henry Fayol dá essa definição no livro General and Industrial Manegement, 1949, e \cite{prasadgulshan2011} diz que ela passa as funções do gerenciamento para se alcançar os resultados desejados. Gerenciar leva você a atingir suas metas, é necessário defini-lás para saber se aquilo que se quer gerir está gerando os resultados esperados, assim como trazer melhorias.

A manutenção, melhor definida no Capítulo~\ref{cap-manutencao}, é aplicar técnicas para manter ou restaurar um ativo para um estado ao qual ele possa realizar as funções requeridas. Gestão da manutenção seria então, controlar, planejar, coordenar a aplicação dessas técnicas para que se possa ter esses ativos gerando resultados positivos e de acordo com o esperado.

Dessa forma, procura-se encontrar nesse trabalho fundamentos para que se possa implementar um gestão da manutenção eficiente na UnB.De acordo com trabalho de \cite{limacastilho2006} que descreve como o DIMEQ funcionava em 2006, ele possui um software, o SIPAT - Sistema de Informações Patrimoniais, que tem como principais funções registrar \emph{acompanhamento do equipamento no período de garantia, cadastro, histórico de manutenção ou procedimento da manutenção e atualização de dados técnicos}. Essas funcionalidades não te dão uma previsão de falhas do equipamento, a criticidade de se ter o equipamento parado por causa da manutenção, o custo de se ter esse equipamento parado e a consequência de sua ausência nas atividades que ele suporta.

O trabalho busca fornecer uma solução informatizada a qual, por meio de indicadores, possa prever quando o equipamento irá falhar, para saber quando realizar a manutenção, os tipos de manutenção que podem ser aplicadas a um certo tipo de equipamento, aquela que possa ser mais eficiente, e assim a partir dos custos de cada opção, auxiliar o gestor na melhor decisão, se é continuar realizando manutenções em um determinado equipamento ou se é melhor considerar a sua substituição. Serão utilizados na solução conceitos como Tempo Médio de Falha, Tempo Médio de Reparo, Tempo Médio Entre Falhas, disponibilidade, confiabilidade e custo. 


%------------------------------------------------------------------------------%

\section{Problema}

Como melhorar a gestão da manutenção de equipamentos eletrônicos na UnB, por meio do desenho de uma solução que informatize-a, e apresente indicadores de desenpenho que auxilie seus gestores na tomada de decisões quanto ao tipo de manutenção que será aplicada a esses equipamentos ou quanto a substuição dele, levando em consideração o custo da escolha. 

%------------------------------------------------------------------------------%

\section{Justificativa}

O trabalho propõe melhorar a gestão da manutenção da UnB, tendo em vista que a má gestão da manutenção desperdiça recursos, trazendo gastos extras e não planejados, e também gera soluções precárias e tardias que elevam a taxa de indisponibilidade dos ativos, como os equipamentos eletroeletrônicos e hospitalares. Além do que a má gestão pode diminuir o tempo de vida útil dos equipamentos tendo no caso particular da UnB, a consequência da diminuição da produção cientifica e da qualidade de ensino.

Em contrapartida, uma melhor gestão da manutenção maximiza a disponibilidade, confiabilidade e segurança dos equipamentos. O que traz maior previsibilidade dos custos, saindo da cultura do \lq\lq quebrou-concerta\rq\rq, atuando-se preventivamente. As falhas inesperadas diminuem fazendo com que o setor de manutenção contribua com o sucesso do negócio e não seja apenas um setor secundário ou de apoio, seja um setor consolidado que contribua ativamente com a melhoria da Universidade.



%------------------------------------------------------------------------------%

\section{Objetivo Geral}
 
O objetivo geral deste trabalho consiste na proposta de uma solução que informatize a gestão da manutenção de equipamentos eletrônicos da UnB, por meio da análise do seu ciclo de vida e da criação de modelos de manutenção, como os mostrados na Seção~\ref{sec_modelos_manutencao}, que possam melhor atender as necessidades de um certo tipo de equipamento. Assim como a realização de um estudao de caso, para validação da solução proposta. 

O estudo de caso consistirá na análise da recente decisão de se substituir o uso de projetores por televisões nas salas de aula da FGA, analisando as características de cada equipamento, seu ciclo de vida, confiabilidade, desempenho e custo.


%------------------------------------------------------------------------------%

\section{Objetivos Específicos}

Os objetivos específicos do trabalho são:

\begin{enumerate}
	\item Identificar falhas na gestão da manutenção de equipamentos eletrônicos da UnB, para sugerir melhorias, voltadas principalmente para a análise do ciclo de vida dos produtos.
	\item Sugerir uma solução que informatize dados importantes para a manutenção da Universidade, e forneça indicadores que auxiliem seus gestores na tomada de decisões.
	\item Simular a aplicação dos indicadores da manutenção por meio de um software interativo de cálculo numérico (MATLAB) 
	\item Validar solução por meio de um estudo de caso.
	\item Apresentar e descrever normas que padronizem a manutenção e gestão de ativos.
\end{enumerate}

%-----------------------------------------------------------%

\chapter{Metodologia}

Esta etapa da pesquisa aborda o objeto de estudo deste trabalho – a gestão das manutenções de Equipamentos Eletrônicos pela Universidade de Brasília, com vistas a conhecer como está estruturada a Diretoria da Manutenção de Equipamentos DIMEQ/UNB e como se dá a operacionalização destas manutenções, por meio de estudos descritivos do organograma da instituição Universidade de Brasília e entrevistas realizadas com os setores da DIMEQ.

Foi realizada uma extensa revisão bibliográfica que é parte de uma pesquisa exploratória, para se obter conhecimento sobre conceitos relativos ao tema, como manutenção, tipos de manutenção, modelos no Capítulo~\ref{cap-manutencao}, bem como gestão de ativos, ativos, sistema de gestão de ativos e normas relacionadas apresentados no Capítulo~\ref{cap-ativos}, softwares de gestão da manutenção Capítulo~\ref{cmms} e medidas básicas de tolerância a falhas, como  Tempo Médio de Falha, Tempo Médio de Reparo, Tempo Médio Entre Falhas, disponibilidade, confiabilidade e custo encontrados no Capítulo~\ref{falhas}.

Assim como a realização de entrevistas para obtenção de informações sobre como são realizadas as manutenções dos equipamentos eletrônicos, quem são os envolvidos, as ferramentas de gestão utilizadas e outras características existentes no processo realizado pelo DIMEQ/UNB.

Foi então comparado o que diz a literatura, para detectar possíveis fragilidades no processo, ao tempo de propor mudanças para informatização do processo e a criação de indicadores de desempenho da manutenção, pela abordagem quantitativa, que é feita por meio de registros e análises dos dados coletados dos equipamentos escolhidos para o estudo. O estudo de caso teve sua realização com base na recente decisão de se trocar os projetores por televisões nas salas de aula da UnB-FGA. 



%---------------------------------------------------------------------------------------------%