\chapter{Introdução}
\label{cap-introducao}

A ABRAMAM, uma organização privada criada em 1984 que trabalha com o objetivo de melhorar o padrão da manutenção no Brasil tanto no seu aspecto técnico quanto na sua gestão, realiza uma pesquisa bienal que busca retratar a situação da manutenção no Brasil,segundo ela existe uma baixo envolvimento das empresas (públicas e privadas), sendo que as participantes são as mais ricas e poderosas e que adotam os mais elevados padrões técnicos  de qualidade de gestão e de segurança. Essa baixa participação resulta em uma pesquisa a qual não reflete um cenário real no Brasil. Esses dados apresentados pela \cite{abraman}, referentes a pesquisas antigas realizadas antes do ano 2000. \textcolor{red}{fechar quando conseguir a pesquisa}

%Estas constatações e demandas por melhorias se tornam relevantes à medida que em um cenário cada vez mais competitivo e exigente, as instituições devem buscar, em todas as suas áreas, a maior eficiência. Assim a manutenção tem a missão de manter funcionando a estrutura produtiva das instituições, para melhorar a produtividade, disponibilidade e confiabilidade dos ativos imobilizados com o menor custo possível.

%

%O ponto inadequado nestas práticas usuais é sua falta de previsibilidade e sua metodologia, que consiste em solicitações manuais de manutenção, que causam atrasos, desorganização, aumento nos custos e consequente perda de produtividade. Uma solução para isto,  aceleraria o processo de  atendimento dessas solicitações a fim de melhorar sua organização. Hoje, existem soluções para problemas parecidos, mas que não apoiam completamente o problema descrito, não dando ao gestor um controle unificado.

%------------------------------------------------------------------------------%

\section{Problema}


  na maioria das instituições e órgãos públicos, o processo de gestão das manutenções corretivas e preventivas,controle dos prazos de garantia, e gestão dos pedidos de manutenção para os equipamentos utilizados pelos usuários (ordens de serviços) são realizados através de processos que utilizam formas manuais de gestão (papéis e planilhas). Dessa forma não existe um meio de controle único que dê ao gestor uma visão geral da situação dos ativos imobilizados empregados em suas atividades, especificando para ele dados como: tempo médio de falha, tempo médio de reparo, porcentagem de equipamentos que carecem de manutenção, quais e quantos equipamentos estão quebrados.


Observou-se a má gestão do processo de manutenção de equipamentos eletrônicos em orgãos e instituições públicas. Falta de controle de equipamentos que estão fora ou dentro da garantia, necessidade de manutenção ou substituição, gastos feitos com manutenção.

%------------------------------------------------------------------------------%

\section{Justificativa}

Tendo como base dados da pesquisa realizada pela \cite{abraman} sobre a situação da manutenção no Brasil, constatou-se que o cenário encontrado na UnB, não se diferencia do cenário nacional. Por isso o trabalho realizado procura melhorar o setor de manutenção da UnB, tendo em vista que a má gestão da manutenção desperdiça recursos, não só financeiros, mas também soluções precárias e tardias que causam mau uso dos ativos, podendo diminuir seu tempo de vida e diminuir a produtividade das atividades suportadas por ativos, como equipamentos eletrônicos, hospitalares, entre outros.

Em contrapartida, uma melhor administração da manutenção maximiza a disponibilidade, confiabilidade e segurança dos equipamentos. O que traz maior previsibilidade dos custos, saindo da cultura do "quebrou-concerta", atuando-se preventivamente. As falhas inesperadas diminuem fazendo com que o setor de manutenção contribua com o sucesso do negócio e não seja apenas um setor secundário ou de apoio, mas algo que faça parte do processo da instituição.


%------------------------------------------------------------------------------%

\section{Objetivo Geral}

O objetivo geral deste trabalho é apresentar uma solução para o aprimoramento do processo de manutenção de equipamentos eletrônicos em orgãos e instituições públicas, por meio da modelagem de uma ferramenta para automatização desse processo. A ferramenta irá dar suporte à gestão da manutenção e na tomada de decisão pelo gestor, de escolher entre a manutenção ou a substituição do equipamento. Ter um ponto de controle único para o gestor sobre a situação dos equipamentos eletrônicos, pode melhorar o funcionamento das atividades nas quais esses são utilizados e dimunuir os valores gastos tanto com manutenção quanto com a substituição dos equipamentos eletrônicos.

%------------------------------------------------------------------------------%

\section{Objetivos Específicos}

Os objetivos específicos do trabalho são:

\begin{enumerate}
	\item Conhecer os atuais modos de gestão da manutenção de equipamentos eletrônicos, a partir de uma revisão bibliográfica e entrevistas realizadas com pessoas de diferentes áreas da manutenção (industrial, hospitalar, predial, e etc).
	\item Identificar falhas na gestão para contribuir na melhoria do gerenciamento dos processos de manutenção de equipamentos eletrônicos, voltados principalmente para a análise do ciclo de vida dos produtos.
	\item Mapear o processo antigo e o novo, para comparar e constatar se foram realizadas melhorias.
	\item Modelar uma ferramenta computacional que possa automatizar o processo de manutenção sugerido.
	\item Realizar um estudo de caso para validar o novo processo sugerido no trabalho.
\end{enumerate}

%







