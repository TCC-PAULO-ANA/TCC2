\chapter{Introdução}
\label{cap-introducao}

A gestão dos ativos (Asset Management) de uma organização, seja ela pública ou privada, é cada vez mais adotada ao redor do mundo como uma ferramenta para enfrentar os desafios econômicos impostos por um mercado globalizado e diversificado. Em relação ao Poder Público, a sociedade exige maior transparência e eficiência no emprego de seus tributos, uma vez que uma administração eficiente tem a capacidade de aperfeiçoar o desempenho técnico e econômico dos ativos, acelerar o retorno sobre os investimentos realizados e colaborar com o planejamento organizacional, trazendo previsibilidade e controle dos gastos envolvidos. 

Dentre as áreas da gestão de ativos, a manutenção é geralmente encarada como uma despesa que se deseja ao máximo adiar, não sendo uma prática suficientemente valorizada. Esta opinião é compartilhada em razão da manutenção ser uma fonte de custo que, aparentemente, não acrescenta um valor perceptível ao cliente final do produto ou serviço e, ainda, gera indisponibilidades momentâneas no uso de bens e recursos. Todavia, inevitavelmente o uso e a a ação do tempo fazem com que equipamentos e instalações se desgastem, necessitando periodicamente de reparos, regulagens e limpezas para que continuem operando eficientemente.

A efetiva gestão da manutenção tem se tornado cada vez mais necessária nas organizações, na medida em que a execução dos processos com eficiência contribuem para a otimização dos recursos e a redução dos desperdícios.  Neste sentido, a correta rotina de manutenções dos ativos e instalações se faz importante porque o custo de se realizar uma manutenção regular pode se tornar muito pequeno quando comparado ao custo de uma interrupção na produção ou atividade.

Na obra The Importance of Maintenance \cite{krar2009}, Steve krar definiu como sendo um dos principais propósitos da manutenção garantir que todos os equipamentos estejam funcionando com 100\% de eficiência o tempo todo, por meio de inspeções diárias e pequenos ajustes, os quais ajudarão na detecção de problemas menores, diminuindo a chance de esses problemas se tornarem maiores ou até incorrigíveis. Krar diz ainda que para se atingir uma manutenção regular e eficaz, é preciso que haja participação de todos, desde o alto executivo até as pessoas do operacional.

Neste contexto se insere o objeto de estudo deste trabalho: a manutenção de equipamentos eletrônicos da Universidade de Brasília – UnB. Esta universidade detém diversos laboratórios de pesquisa e certificação de alto valor e tecnologia, com grande número de aparelhos e equipamentos científicos de sistemas complexos. Todavia, como outras instituições públicas de ensino superior, a UnB adotou políticas para a aquisição de um parque cientifico e tecnológico, mas desconsiderou aspectos necessários ao não verificar a existência de meios humanos e materiais para a correta manutenção deste bens. Além disso, dispõe de processos de gestão de manutenções pouco informatizados, não possuindo indicadores de desempenhos claros que auxiliem seus gestores a tomada de decisão.

Segundo \cite{limacastilho2006}, a falta de manutenção adequada, no caso particular da Universidade de Brasília,  reflete numa taxa de indisponibilidade exagerada dos equipamentos e instalações importantes para os laboratórios de ensino, pesquisa e apoio administrativo, tendo como consequência a diminuição da capacidade produtiva da instituição e a insatisfação daqueles que dependem desse serviço. Assim, a qualidade dos serviços oferecidos pela Diretoria de Manutenção de Equipamentos – DIMEQ, unidade responsável por prover a manutenção e o reparo de equipamentos da Universidade, fica comprometida. 

Portanto, este trabalho tem como escopo fornecer fundamentos e indicadores de desempenho que auxiliem na gestão da manutenção dos equipamentos da Universidade, mais precisamente dos equipamentos eletrônicos, que possuem componentes eletrônicos na maioria dos seus circuitos, tais como: equipamentos biomédicos, de análise clínica, de laboratório, de som e de imagem, e outros de infraestrutura de rede e comunicação.

A manutenção, que será melhor definida no Capítulo~\ref{cap-manutencao}, consiste na aplicação de técnicas para manter ou restaurar um ativo para o estado que ele possa realizar as funções requeridas. A gestão da manutenção seria, então, o controle, o planejamento e a coordenação da aplicação dessas técnicas para que se possa ter ativos gerando resultados positivos e de acordo com o esperado.

Atualmente, a Universidade de Brasília utiliza o Sistema de Informações Patrimoniais (SIPAT) para gerenciar a manutenção de seus ativos, bem como para cadastrar o bem adquirido e o seu histórico de manutenção, registrar o acompanhamento do equipamento no período de garantia e atualizar os dados técnicos. Todavia, tais funcionalidades não permitem prever falhas dos equipamentos, a criticidade de se ter o equipamento parado por causa da manutenção, o custo de se ter esse equipamento parado e a consequência de sua ausência nas atividades que ele suporta. 

Dessa forma, procura-se encontrar nesse trabalho fundamentos para que se possa implementar uma gestão da manutenção eficiente na UnB. Para tanto é proposta uma solução informatizada que por meio de indicadores, possa prever quando o equipamento irá falhar, quando deverá ser realizada a manutenção, quais os tipos de manutenção são adequados e mais eficazes a partir dos custos de cada opção. Espera-se, ainda, que a solução forneça subsídios aos gestores na busca da decisão mais acertada acerca da continuidade da realização de manutenções em um determinado equipamento ou a sua substituição. Serão utilizados na solução conceitos como Tempo Médio de Falha, Tempo Médio de Reparo, Tempo Médio Entre Falhas, Disponibilidade, Confiabilidade, Satisfação do Cliente e Custo. 

Para modelar a proposta foi pesquisado e estudado sobre ferramentas para gerenciamento de manutenção. Encontrando-se então os CMMS, do inglês Computerized Maintenance Management System, que são ferramentas voltadas para gestão da manutenção. Por isso o trabalho busca propor uma ferramenta que possua características comuns a esse tipo de software e com adequações as necessidades do setor de manutenção da UnB.


%------------------------------------------------------------------------------%

\section{Problema}

Falta de um sistema computadorizado, voltado para o setor de manutenção, que auxilie na gestão da manutenção de forma eficaz e também na tomada de decisão, por meio de indicadores de desempenho. Assim como, auxiliar na tomada de decisões quanto ao tipo de manutenção que será aplicada a esses equipamentos ou quanto a sua substituição, levando em consideração o custo da escolha. 


%------------------------------------------------------------------------------%

\section{Justificativa}

O trabalho propõe melhorar a gestão da manutenção da UnB, tendo em vista que a má gestão da manutenção desperdiça recursos, trazendo gastos extras e não planejados, e também gera soluções precárias e tardias que elevam a taxa de indisponibilidade dos ativos, como os equipamentos eletroeletrônicos e hospitalares. Além do que, a má gestão pode diminuir o tempo de vida útil dos equipamentos tendo, no caso particular da UnB, a consequência da diminuição da produção cientifica e da qualidade de ensino.

Em contrapartida, uma melhor gestão da manutenção maximiza a disponibilidade, confiabilidade e segurança dos equipamentos, trazendo maior previsibilidade dos custos e saindo da cultura do \lq\lq quebrou-concerta\rq\rq, atuando-se preventivamente. As falhas inesperadas diminuem, fazendo com que a DIMEQ contribua ativamente com a melhoria da Universidade de Brasília.


%------------------------------------------------------------------------------%

\section{Objetivo Geral}
 
O objetivo geral deste trabalho consiste na proposta de uma solução que informatize a gestão da manutenção de equipamentos da UnB, por meio da análise do seu ciclo de vida e da utilização de tipos de manutenção, que possam melhor atender as necessidades de um certo tipo de equipamento. 

%------------------------------------------------------------------------------%

\section{Objetivos Específicos}

Os objetivos específicos do trabalho são:

\begin{enumerate}
	\item Identificar falhas na gestão da manutenção de equipamentos da UnB, para sugerir melhorias, voltadas principalmente para a análise do ciclo de vida dos produtos.
	\item Apresentar e descrever normas que padronizem a manutenção e gestão de ativos.
	\item Construir indicadores de desempenho da manutenção que avaliem a eficácia das tarefas executadas, dando maior previsibilidade e eficiência aos gastos com manutenção.
	\item Sugerir uma solução que informatize dados importantes para a manutenção da Universidade, e forneça indicadores que auxiliem seus gestores na tomada de decisões.
	\end{enumerate}

%-----------------------------------------------------------%

\section{Metodologia}


Para condução do trabalho, foi adotado o método de pesquisas exploratórias, para que fosse obtida maior familiaridade com o problema a ser resolvido. Segundo, Gil \cite{gil2002elaborar}, esse tipo de pesquisa busca tornar o problema mais explícito, aprimorando ideias e considerando os mais variados aspectos relativos ao fato estudado. 

No trabalho, a pesquisa exploratória envolveu os seguintes procedimentos técnicos:

\begin{itemize}
\item \textbf{Pesquisa Bibliográfica:} Realizada para o aprofundamento dos conceitos envolvidos no tema do trabalho e para suporte teórico na solução proposta.
\item \textbf{Entrevistas:} Realização de entrevistas com pessoas envolvidas no cenário do problema a ser resolvido, para conhecimento da realidade atual  e sugestão de melhorias de acordo com as necessidades dos envolvidos no setor de manutenção.
\end{itemize}

Foi realizada uma extensa revisão bibliográfica, para se obter conhecimento sobre os conceitos relativos ao tema manutenção, seus tipos e modelos (Capítulo~\ref{cap-manutencao}), bem como o sistema de gestão de ativos e normas relacionadas (Capítulo~\ref{cap-ativos}), softwares de gestão da manutenção (Capítulo~\ref{cmms}) e medidas básicas de tolerância a falhas, como  Tempo Médio de Falha, Tempo Médio de Reparo, Tempo Médio Entre Falhas, Disponibilidade, Confiabilidade e Custo (Capítulo~\ref{indicadores}).

Realizou-se, ainda, entrevistas para obtenção de informações sobre como são executadas as manutenções dos equipamentos da Universidade, quem são os envolvidos, as ferramentas de gestão utilizadas e outras características existentes no processo realizado pelo DIMEQ/UNB. Foi então comparado o que diz a literatura, para detectar possíveis fragilidades no processo, ao tempo de propor mudanças para informatização do processo e a criação de indicadores de desempenho da manutenção, pela abordagem quantitativa, que é feita por meio de registros e análises dos dados coletados dos equipamentos escolhidos para o estudo. 

Para a modelagem do sistema proposto foi utilizado como referência o documento de visão, parte da metodologia tradicional RUP, para a definição do produto, seus usuários, suas necessidades, as características de um CMMS levantadas nas pesquisas bibliográficas, assim como o uso da especificação de casos de uso para descrever as funcionalidades do sistema a partir dos requisitos elicitados.

Os requisitos foram levantados após as entrevistas realizadas com os técnicos do CPD da DIMEQ, assim como por meio da técnica de prototipação, utilizada para melhorar a visualização de como seriam as telas do sistema e o fluxo seguido pelo usuário para utilização do sistema. Também foi utilizada a notação de modelagem de processos BPMN (Seção~\ref{bpmn}) para o mapeamento dos processos AS IS (Seção~\ref{processo-as-is}) e TO BE (Seção~\ref{to-be}), sendo o primeiro para visualização de como o processo ocorre atualmente e o segundo uma proposta de processo, com melhorias sugeridas e com embasamento na ferramenta que está sendo proposta.

Para desenho do Diagrama de Caso de Uso, modelagem do processo e dos protótipos das telas do sistema, foi utilizada a ferramenta \textit{draw.io}, escolhida por ser gratuita e pelos vários recursos que possui, além da sua usabilidade.

%---------------------------------------------------------------------------------------------%

\section{Organização do Trabalho}

O trabalho é composto por capítulos que podem ser divididos em três principais partes:

\begin{itemize}
	\item \textbf{Referencial Bibliográfico:} Explanado no Capítulo~\ref{cap-manutencao}, Capítulo~\ref{cap-ativos}, Capítulo~\ref{cmms} e Capítulo~\ref{indicadores}, construídos por meio de uma extensa revisão bibliográfica, com o objetivo de fornecer o embasamento teórico necessário para o entendimento dos conceitos que circundam o tema, e fornecer conhecimento para a construção da solução proposta.
	\item \textbf{Caracterização do Problema:} Exposição do estado atual da manutenção na UnB, sua caracterização, identificação dos perfis envolvidos, e entendimento do processo seguido, no Capítulo~\ref{man-unb}. 
	\item \textbf{Solução Proposta}: Apresentação da solução proposta pelos autores, construída a partir do estudo da atual situação da manutenção de equipamentos na universidade, utilizando-se do conhecimento adquirido na revisão bibliográfica, durante o curso de engenharia de software e eletrônica, no Capítulo~\ref{cap-solucao} e Apêndices.
\end{itemize}

%---------------------------------------------------------------------------------------------%